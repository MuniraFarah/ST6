\section{Problemafgrænsing}
\textit{I dette afsnit opsummeres problemanalysen kort og problemet afgrænses. Det afgrænsede problem defineres i en problemformulering, der danner baggrund for resten af rapporten.}\\
\\
Hjertesvigt er et syndrom karakteriseret ved symptomer og fund, der skyldes strukturelle eller funktionelle abnormaliteter i hjertet. Dette svækker hjertemusklen, så den ikke længere er i stand til at pumpe tilstrækkeligt med blod til kroppens behov.
Da befolkningen bliver ældre, og behandling af akutte hjerte-kar-sygdomme er blevet bedre, er prævalensen stigende. Der er dermed flere, der vil leve med et skadet hjerte længere. %Prævalensen er stigende, da befolkningen bliver ældre, og behandlingen af akutte hjerte-kar-sygdomme er blevet bedre, og der dermed er flere der vil leve med et skadet hjerte efter.
%Noget om mortalitet og genindlæggelser.\\
Hjertesvigt kan inddeles i kronisk og akut, hvor kronisk hjertesvigt udvikler sig over tid med planlagte kontrolbesøg på hospitalet. Der arbejdes derfor videre med kronisk hjertesvigt, da det er ved disse patienter forværring, med hospitalsindlæggelser og øget mortalitet til følge, kan forebygges.\\
Dette kan gøres ved hjemmemonitorering, der har vist positive effekter på antallet af genindlæggelser og mortalitet ved hjertesvigtpatienter. Det er en fordel hvis patienten selv kan anvende udstyret til hjemmemonitorering, så det ikke kræver fagpersonale hver gang der skal tages en måling. Dette udelukker ekkokardiografi og EKG. Det er desuden en fordel hvis udstyret er let tilgængeligt, for at minimere udgifterne til dyrt udstyr til hjertesvigtpatienter. Dette udelukker nogle former for BCG, da erhvervelsen af specielle stole, senge og vægte ikke er hensigtsmæssigt. Det er anslået at PCG kan optages i hjemmet med smartphones med gode mikrofoner. Ofte kræves dog en del filtrering af signalerne. PPG kræver kun en LED og en photodiode, og er der lettilgængeligt, dog er det ekstra udstyr der skal investeres i, i forhold til at bruge udstyr de fleste patienter allerede ejer, som f.eks. en smartphone. Alle smartphones indeholder et accelerometer, og har dermed mulighed for at måle 1g BCG og SCG. Der er dog lavet flere studier af monitorering af hjertesvigt med SCG og dette kræver mindre filtrering end PCG, da kun vejrtrækningen forstyrrer signalet. Der arbejdes derfor videre med SCG.

Da det er muligt at anvende en smartphone til optagelsen af SCG, er det derfor oplagt at lave en app der bruger SCG til hjemmemonitorering af hjertesvigtpatienter. Dette fører til følgende problemformulering:

\subsection{Problemformulering}

\begin{displayquote}
\textit{Hvordan kan en app designes, så den kan anvende seismokardiografi til hjemmemonitorering af patienter med kronisk hjertesvigt?}
\end{displayquote}

