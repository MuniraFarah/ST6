\section{Seismokardiografi}
For hvert hjerteslag produceres et SCG-kompleks, hvis signal har vist sig at korrespondere med specifikke begivenheder i hjertecyklus. Dette indebærer åbning og lukning af aortaklappen, såvel som den hurtige udpumpning af blod ind i aorta \citep{inan2015}. Disse annotationer blev forslået af \citet{crow1994relationship} som et resultat af en direkte sammenligning af SCG med ekkokardiografiske billeder. \\

SCG kan benyttes til at monitorere hjertefunktioner forbundet med et antal hjerte-kar-sygdomme \citep{munir2008}. Blandt andet kan man med SCG udlede hæmodynamiske parametre som kan anvendes til at detektere kardiovaskulære sygdomme, heriblandt kan nævnes systolic time interval (STI) \citep{di2013wearable}. De mest anvendte intervaller som indgår under betegnelsen STI er pre-ejection period (PEP) og left ventricular ejection time (LVET). Disse kan indikere abnormaliteter i hjertemusklens funktion. \citep{Reant2010} Eksempelvis er der hos patienter med hjertesvigt vist en øget PEP og nedsat LVET ift. normal \citep{Marcus2007}.  STI har været et brugbart værktøj i adskillige applikationer såsom at identificere graden af venstre ventrikel muskel dysfunktion, mitralklapsstenose, artrieflimmer, koronararteriesygdom og detektering af iskæmi \citep{Shafiq2016}. \\
En øget PEP er desuden også indikation for en nedsat kontraktilitet. Kontraktilitet definerer hjertemusklens evne til at kontrahere, mens slagvolumen er en indikator for kontraktiliten. Disse to er vigtige aspekter i den kardiovaskulære tilstand som ændrer sig signifikant hos hjertesvigts patienter når deres tilstand forværres. \citep{Ashouri2016}\\

\section{Seismokardiografi til hjemmemonitorering}
SCG som et bærbart målingssystem har den primære fordel at det gøres muligt at opsamle data løbende i dagligdagen, som potentielt kan hentes i ethvert miljø, hvilket muliggør en evaluering af patientens kardiovaskulære præstation \citep{inan2015}. \\ 
SCG til hjertesvigtspatienter er dog en relativ ny metode, som på nuværende tidspunkt ikke er en velintegreret del af sundhedsvæsnet \citep{}. En litteratur gennemgang af SCG som et bærbart system til hjertesvigtspatienter har givet en indikation af dens værdi til måling af disse signaler, men viser ligeledes at veldokumenteret litteratur i dette felt er meget sparsomt. Det var ikke muligt at finde studier, der udelukkende benyttede SCG til at detektere de forskellige parametre, som jf. afsnit \ref{hjertesvigt} gav et udtryk for graden af hjertesvigt. Dette skyldes, at man anvender en kombination af SCG og andre modaliteter til parameterestimering \citep{di2013wearable} \citep{Shafiq2016} \citep{Sahoo2017} \citep{Etemadi2018}. Der er derfor gjort nogle tilnærmelser for at give et indtryk af hvad der på nuværende tidspunkt har en veldokumenteret effekt, hvorfor studier der evaluerede SCG sammen med andre modaliteter blev inkluderet. De inkluderede studier havde desuden forskellige effektmål. \citet{Etemadi2018} udledte hjertemekaniske parametre, herunder STI ved at anvende SCG sensorer og EKG. Robustheden af disse parametre i forskellige fysiske aktiviteter blev undersøgt af \citet{di2013wearable} ved at foretage målinger i løbet af en hverdag, hvor det blev fundet at parametrene var robuste overfor støj. \citet{Shafiq2016} undersøgte effekten af algoritmer til en automatiseret peak detektion af signalet, mens \citet{Sahoo2017} som de andre studier, benyttede SCG som et supplement til det mere udbredte EKG. Sidstnævnte studie lavede en gennemgang af deres system, som blev optaget med sensorer og trådløst overført til en platform til en analyse af signalet. Herefter var målet at generere en alarm ved detektion af hjertemekaniske abnormaliteter. \citet{Inan2018} evaluerede, som det eneste studie, udelukkende på et SCG signal. Dog var målet at estimere graden af kompensation ved at sammenligne SCG før-  og efter en træningssession, hvorfor de ikke undersøgte de enkelte komponenter, men signalet som en helhed. Her bedømte de på ændringen hjertesvigtstilstanden hos ambulante og indlagte patienter med en bærbar SCG og resultaterne indikerede at det var muligt at vurdere hjertesvigttilstanden ved denne fremgangsmåde \citep{Inan2018}. En fællesnævner for disse studier er, at de er udført i et kontrolleret miljø, med andre ord, er der ingen entydige svar på robustheden af SCG under ikke-overvågende miljøer, herunder SCG til hjemmebrug. Derfor kunne fremtidige studier med fordel undersøge SCG til hjemmemonitorering i patienternes eget hjem.
Det er i afsnit \ref{} etableret, at den ældre befolkning, som udgør majoriteten af hjertesvigtspatienter har en dårlig compliance, hvorfor det kunne være relevant at følge patientens tilstand i form af deres respons til farmakologiske interventioner \citep{Inan2018}. Da SCG endnu er på forskningsbasis, udløser det behovet for at undersøge effekten/værdien af denne i patienternes egne rammer, med henblik på at bestemme kvaliteten af SCG som en metode til hjemmemonitorering.
Yderligere, var der ingen af studierne der inddragede SCG målt på en smartphone. Dog indeholder smartphones hardware et accelerometer, hvorfor det antages at kunne bruges i samme grad som simple accelerometre, som de inkluderede studier har vist. Den store tilgængelighed af smartphones idag giver desuden et godt marked for SCG, men dette kræver en nøje gennemgang af SCG på app form.

\section{Problemformulering}
I de første uger efter den den første indlæggelse eller det første ambulante besøg, er det estimeret at 80\% af patienterne oplever forværringer i hjemmet, som medfører genindlæggelser. Hjertesvigtspatienter udgør en stor økonomisk byrde i sundhedssektoren, og alene genindlæggelserne er svarende til over 60\% af denne byrde. På trods af diverse interventioner for at nedsætte mortalitet, morbiditet og genindlæggelser i denne patientgruppe, må det dog konkluderes, at der ikke er sket den markante forskel i patientgruppens fremtidsudsigter i det seneste årti. 
Selvom hjertesvigt er en progressiv lidelse, kan man, med bl.a. en god compliance, nedsætte progressionen og forblive i samme NYHA klasse over en længere periode. Især den ældre del af patientgruppen, som udgør majoriteten, er det estimeret at kun 10 \% har en god compliance. Langt hen ad vejen kan patientuddannelse medvirke til, at patienterne reagerer på forværringen inden de genindlægges, men ofte er symptomerne nonspecifikke, hvorfor de ikke reagerer herpå.

Det er af væsentlig karakter at sætte ind så tidligt i processen som muligt, da forværringer der ikke opfanges i tide kan medføre irreversibel skade på hjertemusklen, bidrage med flere co-morbiditeter og lede til en hurtigere progression af lidelsen. Samtidig er selve genindlæggelsen en prædikter for en dårligere prognose.

SCG til hjemmemonitorering har vist at kunne detektere hjertemekaniske parametre, som kan udlede patientens tilstand løbende i deres eget hjem. Samtidig indeholder smartphones den nødvendige hardware til opfangelsen af SCG signalet, nemlig et accelerometer. Dog forefindes der ingen veldokumenteret litteratur, der evaluerer på værdien af en app baseret SCG til hjemmemonitorering af hjertesvigtspatienter. Dette udløser behovet for at foretage et større studie, der undersøger denne sammenhæng, hvortil der er behov for at designe og implementere et system, der muliggør en indhentning af patientdata til en videre bearbejdning.

Dette har ledt til følgende problemformulering:


\begin{displayquote}
\textit{Hvordan kan en app designes, så den kan anvende seismokardiografi til hjemmemonitorering af patienter med kronisk hjertesvigt?}
\end{displayquote}