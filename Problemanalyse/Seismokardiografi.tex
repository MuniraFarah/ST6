\section{Seismokardiografi}

SCG er en non-invasiv metode udviklet til at optage og analysere hjertets aktivitet som et mål for hjertets sammentrækningsevne. Hjertemuskulaturen og blodets bevægelse producerer vibrationer der transmitteres til brystvæggen som kan måles med et accelerometer. \citep{inan2015} Signalet kan detekteres ved at placere en sensor på det prækordiale område, som regel på xiphoid ved den nederste del af sternum \citep{di2013wearable}. 

For hvert hjerteslag produceres et SCG kompleks, hvis signal har vist sig at korrespondere med specifikke begivenheder i hjertecyklus. Dette indebærer åbning og lukning af aortaklappen, såvel som den hurtige udpumpning af blod ind i aorta \citep{inan2015}. Disse annotationer blev forslået af \citet{crow1994relationship} som et resultat af en direkte sammenligning af SCG med ekkokardiografiske billeder. 

SCG kan udlede hæmodynamiske parametre som kan anvendes til at detektere kardiovaskulære sygdomme, heriblandt kan nævnes systolic time interval (STI) \citep{di2013wearable}. De mest anvendte intervaller som indgår under betegnelsen STI er preeejection period (PEP), left ventricular ejection time (LVET) og electro-mechanical systole (QS2). Disse kan indikere abnormaliteter i hjertemusklens funktion. \citep{Reant2010} Eksempelvis er der hos patienter med hjertesvigt vist en øget PEP og nedsat LVET ift. normalt \citep{Marcus2007}.  STI har været et brugbart værktøj adskillige applikationer såsom at identificere graden af venstre ventrikel muskel dysfunktion, mitralklapsstenose, artrieflimmer, koronararteriesygdom og detektering af iskæmi \citep{Shafiq2016}. 

SCG-systemer kan dermed være i stand til at monitorere hjertefunktioner forbundet med et antal hjerte-karsygdomme \citep{munir2008}. SCG som et bærbart målingsystem har den primære fordel at det gøres muligt at opsamle data løbende i dagligdagen, som potentielt kan hentes i ethvert miljø. Dette gør det muligt at vurdere en persons kardiovaskulære præstation under forskellige miljømæssige forhold \citep{inan2015}. 

Fjernmonitorering af hjertesvigts patienter med bærbare enheder kan potentielt gøre det muligt at foretage justeringer i den enkelte patients behandling, hvilket kan reducere genindlæggelser. I et studie foretaget af \cite{Inan2018} bedømmes hjertesvigtstilstand hos ambulante og indlagte patienter med SCG. Resultaterne af denne undersøgelse indikerer at det er muligt at vurdere hjertesvigts tilstanden ved at sammenligne strukturen af et SCG signal før og efter motion. Målingerne blev foretaget i et kontrolleret miljø, hvortil fremtidige studier bør evaluere denne teknologi i patientens hjem med henblik på hjememmonitorering. Desuden kan fremgangsmåden i fremtiden testes til at følge patientens tilstand og deres respons til til farmakologiske interventioner. \citep{Inan2018} 
 

\section{Noter}



Wearable seismocardiography: Towards a beat-by-beat assessment of
cardiac mechanics in ambulant subjects

Previously, the researchers have annotated the SCG signals, but are limited to manual or semi-automated approaches that require human intervention and is generally quite time-consuming. - taget fra en anden kilde som anvender wearable 




Samtlige biologiske parametre kan udledes fra SCG komplekset; slagvolumen \citep{TavakolianVenter2009}. Derudover kan det estimere systolisk og diastoligsk tids intervaller, som kan give informationer om hjertets funktion. 

Seismocardiographic Changes Associated with
Obstruction of Coronary Blood Flow
During Balloon Angioplasty 



Seismocardiography is a form of displacement cardiography, the general field of techniques that record movement of the body in response to cardiac motion. The concept underlying the use of displacement cardiography for detecting myocardial
ischemia is that ischemia will produce a change in systolic or diastolic left ventricular wall motion that will result in a change in displacement. Other types of displacement cardiography include apexcardiography, kinetocardiography,3 ballistocardiography4,* and cardiokymography.6,7



Myocardial contractility: A seismocardiography approach - ikke peer review
Myokardial kontraktilitet definerer hjertets evne til at kontrahere hjertetmusklen som opnåes ved at myosin og aktin filamenter bindes til hinanden. Samtlige hjerte abnormaliteter påvirker og reducerer myokardiel kontraktilitet. Den gyldne standard til at vurdere myokardiel kontraktilitet er en invasiv metode hvor man ved hjælp af et kateter kan måle ændringen i tryk i venstre ventrikel. 

Slagvolumen er en indikator for myokardiel kontraklitet. Det er forslået at SCG kan estimere slagvolumen [4,5,6]



SEISMOCARDIOGRAPHY: A TECHNIQUE FOR RECORDING
PRECORDIAL ACCELERATION - ikke peer review

SCG indeholder defineret begivenheder associeret med hjertecyklus. Dette blev analyseret ved at sammenligne SCG simultant med M-mode og Doppler ekkokardiogram. Ved fravær af hjertesygdom forbliver SCG målt i hvile stabil over en 3 måneders periode. Der observeres at SCG ændrer sig ved kronisk venstre ventrikel dysfunktion fx ved myokardieinfarkt og dilateret kardiomyopati. Et normalt seismokardiografi er stabil over tid men bliver ændret både under kronisk og akut ændringer i venstre ventrikulær kontraktion. Studiet forslår at seismokardiografi kan være et brugbart værktøj til at detektere og evaluere sygdomme forbundet til venstre ventrikels ydeevne. 


Mechanisms Underlying Isovolumic Contraction and Ejection Peaks in Seismocardiogram Morphology

Det har vist sig at SCG signalet, sammenholdt med  ekkokardiografi som en reference, korresponderer med specifik begivenheder i hjertecyklus. Dette indebærer åbning og lukning af aortaklappen, såvel som den hurtig udpumpning af blod ind i aorta \citep{inan2015}. [4-7] Amplituden og strutkuren i SCG signalet kan blive påvirket af faktorer såsom respiration [7], hjertets kontraktilitet[8] og hjerterytme [9]. 

Systolic Time Intervals and New Measurement Methods
\citep{Tavakolian2016}


Unobtrusive Estimation of Cardiac Contractility and Stroke Volume Changes Using Ballistocardiogram Measurements on a High Bandwidth Force Plate

En øget PEP er indikation for en nedsat kontraktilitet. Kontraktilitet og slagvolumen er to vigtige aspekter i den kardiovaskulære tilstand som ændrer sig signifikant hos hjertesvigts patienter når deres tilstand forværres.  Det ses at slagvolumen nedsættes hos hjertesvigts patienter i tiden før deres tilstand forværres, hvilket skyldes at kontraktilitetn ikke er effektiv. Måling af en patients PEP og slagvolumen udenfor klinikken kan give indsigt i sværhedsgraden af patientens tilstand og kan potential forudsige forværringer, således at lægerne kan gribe ind og justere medicinen for at undgå forværring og genindlæggelse. \citep{Ashouri2016}


[11]





