\chapter{Problemanalyse}
\textit{Dette kapitel har til formål at skabe en grundlæggende forståelse for patientgruppens sygdom, diversitet, forløb og efterforløb. Herudover er der undersøgt hvorledes telemonitorering kan bidrage til et bedre efterforløb for patientgruppen, samt de mulige it-løsninger hertil.}

\section{Hjertesvigt} \label{hjertesvigt}

%De patofysiologiske processer er yderst komplekst. Flere forskellige hypoteser som forklarer tilstanden er fremsat. \citep{Abraham2007}Hjertesvigt forbliver ufuldstændigt forstået af forskere, og der er ikke en enkelt samlet ramme, som har vist sig holdbar \citep{Coronel2001}. 
Hjertesvigt er en progressiv lidelse, som forekommer sekundært til abnormaliteter i hjertemusklens struktur, funktion eller begge. Abnormaliteteten, uanset årsagen, resulterer i at hjertet har en manglende evne til at imødekomme kroppens metaboliske behov \citep{Shah2011}\citep{Fletcher2001}\citep{Francis1998}\citep{Mudd2008}. Dette syndrom er associeret med nedsat motionstolerance, dyspnø under hvile og anstrengelse og ødemer i underekstremitet \citep{Francis1998}. Derudover ses fund som øget halsvenetryk, krepetitioner i lungerne og perifere ødemer \citep{heartfailure}.\\
Hjertesvigt er et almindeligt klinisk syndrom, men de patofysiologiske faktorer kan variere betydeligt mellem patienterne \citep{Parmley1985}. Selvom der er mange årsager eller tilstande som kan føre til hjertesvigt, så er de mest almindelige årsager ukontrolleret hypertension, koronararteriesygdom og mitral eller aortaklap dysfunktion \citep{Fletcher2001}. Derudover ses der også sygdomme som kronisk øger belastningen på hjertet, som ved tab af myocytter pga. hjerteinfarkt \citep{Shah2011}. \\
Et normalt fungerende hjerte er i stand til at foretage præcise justeringer i slagvolumen til at imødekomme ændringer i kroppens metaboliske behov ved hvile og motion. Disse fysiologiske variationer i slagvolumen er mulige grundet hjertets elasticitet og er sammen med minutvolumen påvirket af fire variable: inotropi eller hjertets kontraktilitet, preload eller diastolisk fyldningsvolume, afterload eller mængden af tryk som ventriklen skal overkomme for at aortaklappen åbnes, og kronotropi eller hjerterytme \citep{Fletcher2001}.  \\
Overordnet findes to typer af hjertesvigt: systolisk og diastolisk. Forskellen mellem systolisk og diastolisk dysfunktion kan bestemmes vha. ejection fraction (EF). Systolisk dysfunktion er karakteriseret ved EF mindre end 40 \% \citep{Consensus1999} \citep{Fletcher2001}. Systolisk dysfunktion er karakteriseret ved en reduceret venstre ventrikel kontraktilitet, der ofte resulterer i et dilateret hjerte, som ikke er i stand til at vedligeholde en tilstrækkelig minutvolumen \citep{Mudd2008} \citep{Consensus1999} \citep{Fletcher2001}. Hos patienter med diastolisk hjertesvigt, er volumen i den venstre ventrikel typisk normal, hvor EF er større end 50 \% \citep{Ruzumna1996},  og oftest ses hypertrofisk kardiomyopati \citep{Braunwald2013}. Hjertet er i dette tilfælde ikke i stand til at fylde sig tilstrækkeligt under diastole, hvilket fører til en nedsat slagvolumen, nedsat minutvolumen og symptomer på hjertesvigt \citep{Ruzumna1996} \citep{Willerson1995}\citep{Fletcher2001}. Generelt er det iskæmiske hjertesygdomme, som er den mest almindelige årsag til systolisk dysfunktion, mens venstre ventrikel hypertrofi, sekundært til hypertension eller klap abnormaliteter, udgør størstedelen ved diastolisk dysfunktion \citep{Mcalister1997}.\\
Når der er en nedsat minutvolumen og deraf nedsat gennemsnitligt arterietryk, er der en stimulering af flere neurohormonelle systemer som vedligeholder hæmodynamisk stabilitet \citep{Mccance1998}. Her kan kan blandt andet nævnes en stimulering af det sympatiske nervesysstem (SNS) til at udskille adrenalin og noradrenalin, som medfører en øget perifer vaskulær modstand, hjerterytme og kontraktilitet \citep{Fletcher2001}. Redistribuering af blodflow, som resultat af SNS stimulering, nedsætter renal perfusion, som medfører en aktivering af renin-angiotensin-aldosteron-systemet (RAAS). Dette vil efterfølgende medføre vasokonstriktion og retention af vand og salt \citep{Mccance1998}. Renal retention af salt og vand sammen med en øget perifer vaskulær modstand fører til en øget preload og afterload, som medvirker til pulmonale og vaskulære ødemer og skaber symptomer som er karakteristisk ved hjertesvigt \citep{Mccance1998}.\\
En aktiverering af SNS og RAAS vil på længere sigt medføre remodellering af ventriklerne og yderligere myokardieskader, hvorved en ond cirkel opstår \citep{Braunwald2013}. Ventrikulær remodellering er en kompleks patofysiologisk proces som manifesterer sig som ændringer i størrelse, form og funktion af hjertet. Eksempelvis vil en skade på hjertet medføre myocytnekrose \citep{Cohn2000}. Dette ses blandt andet hos patienter med koronararteriesygdom, som har haft en eller flere myokardieinfarkter \citep{Page1971}. Som et forsøg på at vedligholde minutvolumen efter tab af kontraktilt væv, vil de overlevende myocytter forlænge og hypertrofere. Som ventriklerne forstørres vil ventiklernes vægge blive tyndere og begynde at dilatere, hvilket resulterer i dilateret kardiomyopati \citep{Cohn2000}. Remodellering kan forekomme efter akut hjerteinfarkt eller globalt som resultat af kardiomyopati. Hvis det ikke behandles, kan ændringen i venstre ventrikels geometri resultere i ændringer i muskelfunktionen og kan føre til udvikling af sekundær mitral insufficiens, som efterfølgende fører til yderligere forringelse af minutvolumen \citep{Gheorghiade1998}.\\
Initielt er disse kompensatoriske mekanismer gavnlige, men over tid vil disse dog stresse hjertet yderligere og forværre hjertesvigt \citep{Mccance1998} som i sidste ende øger sandsynligheden for organsvigt eller en forværring i den klinisk prognose\citep{Mudd2008}. Her kan hjerterytmen nævnes, der som regel er øget ved hjertesvigt, som formentlig afspejler en stigning af cirkulerende katekolaminer og sympatisk tonus. Stigning i hjerterytmen er vigtigt for at vedligeholde minutvolumen ved lav slagvolumen. Under visse omstændigheder, kan overdrevet  hjerterytme være skadeligt. Eksempelvis er hjerterytmen proportional med hjertets iltforbrug, hvorfor en øget hjerterytme øger hjertets iltbehov. Samtidig er der en invers relation mellem en øget hjerterytme og den diastoliske tid, hvilket giver en mindre blodforsyning til hjertet. Derfor kan en stigning i hjerterytme ikke kun øge efterspørgslen, men kan potentielt reducere forsyningen, og kan dermed have en negativ effekt for patienten. \citep{Parmley1985}\\
Afhængigt af om der er tale om en kronisk eller akut tilstand af sygdommen, med andre ord om symptomerne manifesterer sig gradvist eller om de opstår øjeblikkeligt, følger begge patienttyper det samme forløb. Dette skyldes, at den kroniske patient kan opleve akutte episoder og dermed kan begge tilstande altså have behov for øjeblikkelig behandling.
%Da hjertesvigt er en progressiv lidelse, vil en sen diagnose medføre en øget mortalitet \citep{Veldhuisen1995}. 

\section{Hjertesvigtsforløbet}
Hjertesvigtforløbet kan, ifølge \citet{Gheorghiade2009}, inddeles i følgende fire faser: en tidlig fase, en hospitalsfase, en præ-udskrivelsesfase, og en tidlig post-udskrivelsesfase, hvoraf sidstnævnte varer de første få uger efter udskrivelsen. Dog er det ikke alle patienter der indlægges efter den tidlige fase, idet nogle af dem ikke indstilles til en indlæggelse.
\subsection{Den tidlige fase}\label{dentidligefase}
Den tidlige fase foregår som regel på skadestuen, hvor patienten stabiliseres og årsagen til sygdommen identificeres og behandles. Her foretages en række undersøgelser med henblik på at stille diagnosen \citep{heartfailure}. Der bliver altid taget et elektrokardiogram, EKG, der benyttes til at styrke mistanken om hjertesvigt. Resultatet kan desuden udelukke akutte tilstande som akut hjerteinfarkt. Derudover er ekkokardiografi essentielt i diagnosen af hjertesvigt, da undersøgelsen giver en visualisering af hjertet, hvor tegn på blodprop og hjerteklapfejl kan ses og pumpefunktion kan vurderes. \citep{heartfailure} \citep{DCS}\\
% Derfor er det vigtigt, at patienter, i løbet af behandlingsforløbet, går til rutinemæssig kontrol hos egen læge for at sikre at sygdommen ikke forværres og at patienten dermed får den rette behandling \citep{heartfailure}\\
Der benyttes flere klassificeringsmetoder til hjertesvigt, der er baseret på enten symptomer eller progression \citep{heartfailure}. I Danmark sigtes der efter at klassificere minimum 90\% af samtlige hjertesvigt  ud fra New York Heart Association (NYHA) \citep{RKKP2017}. NYHA består af 4 funktionsklasser (Tabel \ref{tab:NYHA}), hvoraf disse er kendetegnet ved en graddeling af symptomer. NYHA I svarer til første stadie, hvor der ikke opleves symptomer ved hverken hvile, lettere- eller moderat fysisk aktivitet. NYHA II er symptomfri under hvile og ved lettere fysisk aktivitet, men oplever åndenød, træthed og hjertebanken ved moderat fysisk aktivitet. NYHA III har ingen symptomer i hvile, men ved lettere fysisk aktivitet som påklædning eller gang i fladt terræn giver udmattelse, åndenød og evt. hjertebanken eller brystsmerter. NYHA IV opleves der symptomer i hvile, og forværres ved fysisk aktivitet.\\

\begin{table}[H]
\centering
\begin{tabular}{|l|c|c|c|c|}
\hline
\multirow{2}{*}{NYHA klasse} & \multicolumn{3}{c|}{\begin{tabular}[c]{@{}c@{}}Symptomer\\ under aktivitet\end{tabular}} & \multicolumn{1}{l|}{Prædiktiv prognose} \\ \cline{2-5} 
                             & \multicolumn{1}{l|}{Hvile}   & \multicolumn{1}{l|}{Let}  & \multicolumn{1}{l|}{Moderat}  & \multicolumn{1}{l|}{1 års mortalitet}   \\ \hline
NYHA I                       &                              &                           &                               & 5 - 10 \%                               \\ \hline
NYHA II                      & $\times$                             &                           &                               & 10 - 20 \%                              \\ \hline
NYHA III                     & $\times$                             & $\times$                          &                               & 30 \%                                   \\ \hline
NYHA IV                      & $\times$                             & $\times$                         & $\times$                              & 50 \%                                   \\ \hline
\end{tabular}
\caption{En oversigt over NYHA's fire funktionsklasser samt deres respektive prædiktive prognoser. Modificeret fra \citep{DCS} \citep{EdokHjertesvigt} \citep{sundhedprognoser}}
\label{tab:NYHA}
\end{table}

Endvidere fungerer NYHA-klasserne som en prognostisk indikator \citep{Hjerteinsufficiens}, hvor prognosen for funktionsklasserne I til IV er hhv. 5-10 \%, 10-20 \%, 30 \% og 50 \% (Tabel \ref{tab:NYHA})\citep{sundhedprognoser}. En stor andel af patienterne oplever desuden at forblive i samme funktionsklasse over en lang række år, mens det menes at medicin bidrager til at patienten kan vende tilbage til lavere niveau \citep{Hjerteinsufficiens}.  
Klassificeringen af hjertesvigt udgør derfor fundamentet for prognosen og deraf behandlingsvalget \citep{DCS} \citep{EdokHjertesvigt}. Dog afhænger en god prognose i høj grad af en tidlig diagnose, en god compliance, rehabiliteringsforløb, den bagvedliggende årsag samt komorbiditeten, da patienter med hjertesvigt typisk præsenteres med et kompleks af årsager \citep{sstpakke}. Af denne grund individualiseres behandlingen af patienterne \citep{TSchroeder2016} \citep{sstpakke}. Som et led i behandlingen tilbydes patienterne både en farmakologisk og en non-farmakologisk behandling, afhængigt af sygdommens karakter. Formålet er at eliminere eller forhindre progression af lidelsen samt øge livskvaliteten \citep{sstpakke}.\\
Den non-farmakologiske behandling består af modifikation af risikofaktorerne ved livsstilsændringer, herunder rygeophør, vægttab, motion samt diabetes og hypertension \citep{Hjerteinsufficiens}.\\
Den farmakologisk behandling kan være en kombination af forskellige præparater og består typisk af ACE-hæmmere, diuretica, beta-blokkere og aldosteron antagonister \citep{TSchroeder2016}\citep{sstpakke}. Nogle patienter henvises til en mere invasiv revaskularisering, i form af perkutane metoder (PCI) eller by-pass-operation \citep{sstpakke}. Patienter i NYHA klasserne III og IV har ligeledes gavn af en elektromekanisk behandling i form af en Implantable Cardioverter Defibrillator (ICD). Herudover overvejes hjertetransplantation til patienter under 60 år, med terminal hjertesvigt \citep{EdokHjertesvigt}\citep{Hjerteinsufficiens}.

\subsection{Hospitalsfasen}
I hospitalsfasen er patienten indlagt og behandles og monitoreres for eventuelle hjerteskader \citep{Gheorghiade2009}.


\subsection{Præ-udskrivelsesfasen} \label{praudskrivelse}
Ved præ-udskrivelsesfasen oplever størstedelen af patienterne at have fået en lindring i symptomerne, at den bagvedliggende årsag til anfaldet er håndteret, samt at patienten har fået en plan om efterbehandlingsforløbet med klare instruktioner herom \citep{Gheorghiade2009}. Det er ligeledes i denne fase, at opfølgningen planlægges. Ifølge amerikanske og europæiske retningslinjer, bør den første lægekonsultation efter udskrivelsen finde sted efter 1 til 2 uger \citep{Yancy2013}. I Danmark bliver der, ifølge \citet{Sundhedsstyrelsen2018}, foretaget en systematisk behovsvurdering af samtlige patienter med henblik på at planlægge opfølgningen, hvor datoen højst må sættes 2 uger efter udskrivelsen. Til konsultationen vil der ske en vurdering af patientens sygdomsstadie, compliance, rehabilitering og en kontrol af risikofaktorerne blodtryk, kolesterol, og blodsukker \citep{Hjerteinsufficiens} \citep{EdokHjertesvigt}.

\subsection{Post-udskrivelsesfasen} \label{postudskrivelse}
Post-udskrivelsesfasen er der hvor majoriteten af patienterne, oplever en forværring af deres symptomer og renale funktioner grundet hæmodynaiske og neurale abnormaliteter (Afsnit \ref{hjertesvigt}), hvilket er medvirkende til den høje mortalitets- og genindlæggelsesrate \citep{Gheorghiade2009}. Disse er på henholdsvis 10 \% og 20 \% efter udskrivelsen, men stiger til henholdsvis 20 \% og 30 \% efter 3-6 måneder \citep{GFonarow2007}.
Ifølge \citet{Keenan2008} genindlægges hver fjerde patient efter 30 dage og næsten 50 \% af genindlæggelserne er hjerte-relaterede \citep{Gheorghiade2009} \citep{Inan2018}, mens det i Danmark er hver 12. der genindlægges efter de første 4 uger \citep{RKKP2017}. Fælles er dog, at der vil være en stigende tendens i disse fund, da behandlingen af akutte hjerte-kar-sygdomme er blevet bedre \citep{heartfailure} \citep{Gheorghiade2009}.

Forværringen, som opstår på trods af at patienten er under behandling, kan tilskrives en lang række faktorer. Herunder dårlig compliance, dårlig patientuddannelse, eller udløsende faktorer, såsom iskæmi, hypertension og atrieflimmer \citep{Gheorghiade2009}. Disse medfører en stigning i frekvensen af hjertesvigtsrelaterede genindlæggelser \citep{Murray2009}. I et studie foretaget af \citet{Michalsen1998}, blev det estimeret, at over halvdelen af genindlæggelserne potentielt kunne afværges, idet hovedparten af disse vurderedes at være compliance relaterede \citep{Michalsen1998} \citep{Hjerteinsufficiens}. Desuden er selve ndlæggelserne medvirkende til en dårligere prognose samt en prædiktiv indikator for en senere genindlæggelse \citep{Gheorghiade2009}, hvorfor disse bør forebygges.

\subsection{Forebyggelse af genindlæggelser} \label{forebyggelse}
Der er, i litteraturen, konsensus om, at uddannelse, opfølgning og rehabilitering af patienten er væsentlige for at bedre prognosen og reducere antallet af genindlæggelser \citep{kilde}. Compliance er desuden især en udfordring i den ældre del af befolkningen, hvor det ifølge \citet{Murray2009} kun er 10 \% der tager deres medicin som foreskrevet. Samtidig er en kontrol af risikofaktorerne af væsentlig karakter, da disse katalyserer forværringen \citep{sstpakke}. Der er gjort en stor indsats i at forbedre morbiditet og mortalitet i denne patientgruppe over de seneste årtier \citep{VBetihavas2013}. Disse tiltag indgår på nuværende tidspunkt i de amerikanske og europæiske retningslinjer for hjertesvigt \citep{Yancy2013}. I Danmark er hjertesvigtsforløbet desuden tilrettelagt som et pakkeforløb, for at sikre en effektiv behandling og rehabilitering af patienten \citep{sstpakke} og dermed reducere mortalitet og morbiditeten. Til trods for dette, er hjertesvigt en af de mest omkostningsrige lidelser, som udgør 2 \% af samtlige omkostninger i sundhedssektoren, og her er indlæggelserne skyld i to-tredje dele af disse omkostninger \citep{Shafie2018}. Dette skyldes blandt andet patienternes komorbiditet, alder \citep{JOyanguren2016}, men også at der ikke reageres på forværringen i tide, da advarselssignalerne og symptomerne oftest er uspecifikke eller ikke indset af patienten selv \citep{VConraads2011}, hvilket i sidste ende medfører genindlæggelserne. Forværringen er ligeledes medvirkende til at der opstår hjerteskader og en nedsat renal funktion (Afsnit \ref{hjertesvigt}), hvilket er bidragende faktorer til prognosen \citep{Gheorghiade2009}. Desuden er der en sammenhæng mellem antallet af episoder med forværring samt genindlæggelsen og en dårligere prognose \citep{JOyanguren2016}\citep{VConraads2011}\citep{VLueder2012}, særligt i den ældre del af befolkningen \citep{VLueder2012}.
Foruden de hjerterelaterede konsekvenser for den ældre befolkning, medfører indlæggelserne også ikke-hjerterelaterede komplikationer, herunder fald, infektioner og konfusion \citep{VBetihavas2013}.

Det er altså estimeret, at 8,3 \% af patienterne med hjertesvigt genindlægges grundet forværringer der opstår 
kort tid efter den første udskrivelse fra hospitalet (Afsnit \ref{indledningen}) og at 50 \% af disse genindlæggelser er hjertesvigtsrelaterede. Eftersom forværringen over tid, blandt andet, giver anledning til irreversible skader på hjertet, en dårligere prognose, en højere genindlæggelsesrate og flere omkostninger, er der brug for at patienterne reagerer tidligst muligt på forværringerne, hvilket er forsøgt igennem patientuddannelse. Dog viser det sig, at genindlæggelsesraten har været relativ konstant igennem det seneste årti \citep{Inan2018}, hvorfor der er behov for en indsats så snart de første tegn viser sig. Da forværringen sker i patientens eget hjem, er en form for hjemmemonitorering en oplagt mulighed for at sætte ind så snart der varsler tegn på forværring og ihvertfald inden forværringen har forvoldt irreversibel skade.