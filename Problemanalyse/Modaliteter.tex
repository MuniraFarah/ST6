\section{Modaliteter til monitorering af hjertesvigt}
\textit{}
Der er forskellige måder at visualisere hjertets funktionsevne. Som nævnt i afsnit \ref{dentidligefase}, bruges både EKG og ekkokardiografi klinisk til at vurdere om der er hjertesvigt. Udover disse modaliteter findes der andre der kan bruges til at vurdere hjertets pumpeevne.\\
I det følgende er undersøgt forskellige modaliteter til vurdering af hjertefunktionen, der også er mulige at bruge til hjemmemonitorering. Derfor er modaliteter som ekkokardiografi og cardiac-MRI ikke medtaget, da dette kræver specialudstyr og uddannet personale til betjening. De modaliteter der er undersøgt er EKG, der i Danmark bruges til bl.a. diganosticering af hjertesvigt, og derudover er der undersøgt ballistokardiografi, fonokardiografi, photoplethysmografi og seismokardiografi, der alle er non-invasive og mulige at lave i hjemmet \citep{DCS} \citep{heartfailure} \citep{inan2015} \citep{jain2014}.
Det vil blive beskrevet for disse forskellige modaliteter, hvad de måler, hvor de bruges, hvad de præcis kan hjælpe med at diagnosticere, hvilken hardware de kræver, og hvorfor de eventuelt ikke er i klinisk anvendelse endnu.

%\subsubsection{Ekkokardiografi}
%Ekkokardiografi er en form for ultralyd, og den mest brugte modalitet til billeddannelse af hjertet. Det kan bruges til at evaluere både strukturen og funktionen af hjertet, og kan tages enten gennem brystkassen (transtorakal) eller gennem spiserøret (transesophageal). Transesophageal ekkokardiografi giver en bedre billedkvalitet, men indebærer en lille risiko for komplikationer. Ved hjælp af dopplereffekten er det muligt også at bruge ekkokardiografi, og anden ultralyd, til at måle flow i f.eks. hjertekamre. \citep{Matthias2014}\\
%Hvad måler det?
%Den mest almindelige grund til at lave transtorakal ekkokardigrafi er for at vurdere funktionen af venstre ventrikel. Ved at scanne venstre ventrikel fra forskellige vinkler, kan den systoliske funktion blive evalueret. Det er desuden muligt at se bevægelsen af væggen i venstre ventrikel, og dermed se om der er iskæmi eller arvæv, der forhindrer en optimal funktion. Diastolisk funktion kan også vurderes, og gør det dermed muligt at skelne mellem systolisk og diastolisk hjertesvigt. Transesophageal ekkokardiografi bruges hvis billedkvaliteten af transtorakal ekkokardiografi ikke giver nok information om de strukturer der ønskes set. \citep{Matthias2014}\\
%Som tidligere nævnt bruges ekkokardiografi til at bekræfte diagnosen hjertesvigt, og til at skelne mellem diastolisk og systolisk hjertesvigt. Dette gøres på sygehuset. En af fordelene ved ekkokardiografi, er at det ikke er invasivt og ikke indebærer radioaktiv stråling. Dog kræver
%ekkokardiografiundersøgelsen, både transtorakal og transesophageal, en erfaren læge eller radiograf, både til udførelsen af undersøgelsen og fortolkning af billederne. \citep{Matthias2014}

\subsubsection{Elektrokardiografi}
EKG er en metode hvorpå det er muligt at måle den elektriske aktivitet fra hjertet. EKG måles gennem elektroder placeret på overfladen af huden, og er en af de mere hyppigt anvendte metoder til diagnosticering af hjertesygdomme. Et normalt-udseende elektrokardiogram bruges til udelukkelse af hjertesvigt, hvor mindre end 2\% af patienter med normalt EKG har hjertesvigt \citep{authors2012esc}. Dog er et abnormt elektrokardiogram ikke nødvendigvis en specifik indikator for hjertesvigt. Alligevel er EKG’et brugbart, da det som diagnostisk værktøj angiver problemer med hjertet, disse ofte forårsaget af hjertesvigt \citep{davie1996value} \citep{authors2012esc} \citep{madias2011recording}.\\
EKG’et bruges til diagnosticeringen af hjertets elektriske signaler, og abnorme EKG’er i form af ændrede P-bølger, Q-bølger, ST-segmenter og/eller T-bølger, kan være en indikator for hjertesvigt \citep{madias2011recording}. Abnorme EKG’er kan indebære, men er ikke begrænset til: venstresidig ventrikelhypertrofi, atrieflimmer, myokardieinfarkt \citep{davie1996value}.\\
Selvom EKG ikke er invasivt og bruges klinisk til at hjælpe med at stille diagnosen hjertesvigt, kræver det elektroder der skal påklistres huden hver gang der skal tages en måling. Placeringen af elektroder er desuden vigtig, og det er derfor ikke praktisk at anvende til hjemmemonitorering.

\subsubsection{Ballistokardiografi}
Der sker ændringer i kroppens tyngdepunkt, når blodet bevæger sig gennem blodårene ved hvert hjerteslag. Måling af dette kaldes ballistokardiografi (BCG), og kan måles som både forskydning, hastighed eller acceleration i tre akser. BCG kan bruges til at vurdere hjerte-kar-funktionen, da det netop måler ændringer som hjertets output er skyld i. \citep{inan2015}\\
Ballistokardiografi bruges på nuværende tidspunkt ikke til hjertesvigtpatienter på sygehusene i Danmark \citep{DCS}. Det er dermed kun i brug i forbindelse med forskning. En af grundene til dette, er at der er store forskelle i BCG-signal fra person til person, hvilket gør det svært at kvantificere \citep{inan2009}.\\
Der er overordnet fem typer af moderne BCG-måleudstyr, der alle har sine begrænsninger i forhold til hvilke akser de kan måle, og hvor gode signalerne bliver \citep{inan2015}:
\begin{itemize}
    \item 0g accelerometer, der kræver reduceret tyngdekraft, men kan måle i alle tre akser.
    \item 1g accelerometer, der kan måle i aksen hoved til fod, men hvor målingerne varierer med hvor accelerometeret er placeret, og hvor meget forsøgspersonen bevæger sig.
    \item En speciel seng, der enten kan måle i aksen hoved til fod, eller i aksen dorsal til ventral, hvor ændringer i soveposition kan påvirke kvaliteten af signaler.
    \item En stol der ligesom sengen kan måle enten hoved til fod eller dorsalt til ventralt, men hvor signalet og reproducerbarhed er påvirket af positur af forsøgspersonen.
    \item En vægt der kan måle i aksen hoved til fod, men hvor posituren igen påvirker signalet og reproducerbarheden. 
\end{itemize}

Til hjemmemonitorering er det upraktisk at skulle anskaffe nyt og dyrt udstyr til patienterne, hvorfor BCG med specielle vægte, stole og senge ikke er en mulighed. 0g accelerometeret kræver reduceret tyngdekraft, hvilket heller ikke er en mulighed ved hjemmemonitorering. Ved 1g accelerometer vil målingen varriere med hvor accelerometeret er placeret på patientens leje, og det kan derfor være besværligt at placere accelerometeret samme sted hver gang.

\subsubsection{Fonokardiografi}
Fonokardiografi (PCG) baseres på de lyde der kan høres ved forskellige mekaniske events i hjertet. Der er fire primære hjertelyde: S1 og S2, som angives de fundamentale hjertelyde, samt S3 og S4 der høres i mere sjældne tilfælde. S1 opstår når mitral- og tricuspidal-klappen lukker, og S2 er lyden når pulmonal- og aorta-klapperne lukker. S1 og S2 kan dermed angives som hjerteklap-lyde \citep{kovacs2011fetal}. S3 identificeres som vibration af ventriklerne, og S4 som kontraktion af atrier, begge ved fyldning af ventriklerne. \citep{singh2013heart}\\
S3 og S4 anses, sammen med den systoliske tid for hjertecyklussen, som indikatorer for venstresidig ventrikulær dysfunktion, og kan hermed relateres til hjertesvigt \citep{marcus2005association}\citep{shapiro2007diagnostic}.\\
Da PCG måler hjertets lyde, er der behov for en mikrofon til udførelse af målingerne. Måling af S1 udføres ud for mitral- og tricuspidal-klappens placering, og for S2 ud for pulmonal- og aorta-klappens placering. Disse målinger kan udføres med en mikrofon indbygget i et stetoskop, og der er derfor ikke behov for dyrt udstyr. Optagelse af PCG kan dermed udføres i klinikken, og det anslås yderligere af \citet{singh2013heart}, at det kan være muligt med brug af mobiler med gode mikrofoner, at udføre PCG-målinger \citep{singh2013heart}. PCG-signaler er dog meget følsomme for støj fra omgivelserne, bevægelse af mikrofonen og vejrtrækning, hvilket ofte gør en del filtrering af signalet nødvendig \citep{jain2014}.\\

\subsubsection{Photoplethysmografi}
Ved hvert hjerteslag vil blodårene udvide sig og trække sig sammen. Dette kan måles ved at sende lys med forskellige bølgelængder, vha. en LED, gennem huden, f.eks. ved øreflippen eller pegefingeren. Intensiteten af det reflekterede lys, som opfanges af en photodiode, vil ændre sig over tid som følge af hjerteslagene, og målingen af dette kaldes photoplethysmografi (PPG). \citep{jain2014}\\
PPG kan give informationer om puls, og er brugt i vid udstrækning på hospitaler til netop dette, ved at anvende PPG på pegefingeren \citep{jain2014}. Dette giver desuden også informationer om iltmætningen \citep{jain2014}. Et studie af \citet{Gilotra2016} viste, at en højere amplitude af PPG-signalet under en Valsalvamanøvre relativt til baseline, er højere ved højere fyldningstryk i hjertet.\\
En af fordelene ved PPG, er at udstyret er lettilgængeligt og ikke vejer meget, hvilket gør det brugbart til monitorering over længere tid \citep{jain2014}. Det er dog ikke hensigtsmæssigt hvis patienten skal udføre valsalvamanøvren selv hver gang der skal tages en måling, da det at hæve blodtrykket indebærer en risiko for patienten, f.eks. hvis denne har en aneurisme.

\subsubsection{Seismokardiografi}
Hjertemuskulaturen og blodets bevægelse producerer vibrationer, der transmitteres til brystvæggen, og dér kan måles med et accelerometer på det prækordiale område \citep{inan2015}\citep{Salerno1990}, som regel på xiphoid ved den nederste del af sternum \citep{di2013wearable}. Dette er betegnet seismokardiografi (SCG).\\
SCG er i litteraturen første gang nævnt i et klinisk studie af \citet{Salerno1990}, hvor SCG blev foreslået som et brugbart værktøj til at detektere og evaluere sygdomme forbundet til venstre ventrikels ydeevne. Studiet observerede at et SCG signal er stabil over en 3 måneders periode, men bliver ændret ved kronisk venstre ventrikel dysfunktion, f.eks. efter myokardieinfarkt og dilateret kardiomyopati \citep{Salerno1990}.\\
Det er forslået at SCG kan estimere slagvolumen \citep{McKay1999}. Det ses at slagvolumen nedsættes hos hjertesvigts patienter i tiden før deres tilstand forværres, hvilket skyldes at kontraktiliteten ikke er effektiv. Måling af en patients slagvolumen kan dermed give indsigt i sværhedsgraden af patientens tilstand og kan potentielt forudsige forværringer, således kan lægerne intervenere og dermed undgå forværring. \citep{Ashouri2016}\\
Det eneste der kræves for at måle SCG er et accelerometer, og et sådant er tilgængeligt i alle smartphones. Det er derfor hardware mange patienter i forvejen har adgang til. SCG er følsomt overfor bevægelse af patienten, da accelerometeret skal ligge på brystkassen ved måling, men dette kræver mindre filtrering end PCG. Der arbejdes derfor videre med SCG som en modalitet til telemonitorering af hjertesvigtpatienter.

% Tabel over modaliteterne (parametre: hvad måler de, hvad bruger de, hvad koster de)
% BCG og SCG er kandidater..
% Afgræns til SCG her.. Derfor undersøges hvor udbredt denne teknologi er
%*** smartphones (med deres acc og gyro) er veldokumenterert i litteraturen.....
