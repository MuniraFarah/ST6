\section{Telemonitorering}

Telemonitorering er brugen af telekommunikation til at levere sundhedsydelser på afstand. Dette er gjort muligt, da adgangen til telekommunikationsudstyr som mobiltelefoner og computere er øget.
Denne teknologi kan overordnet deles op i tre kategorier: fjernmonitorering, store-and-forward og interaktiv telemedicin. Disse kan desuden kombineres. \citep{InteractiveTelemedicine}

\textbf{Fjernmonitorering} er teknologier som mobiltelefoner, der kan bruges til at monitorere patienter med langvarige sygdomme, ved at overføre kliniske data fra patienten derhjemme til en læge. I forhold til rutinemæssige ambulante besøg, gør dette lægen i stand til hurtigere at ændre behandlingen, hvis patientens data giver grundlag for dette. \citep{InteractiveTelemedicine}

\textbf{Store-and-forward} systemer gemmer data, så det kan blive analyseret på et senere tidspunkt. Dette kan f.eks. være billeder taget af en radiograf, der gemmes elektronisk, og senere kan tilgås af en læge. \citep{InteractiveTelemedicine}

\textbf{Interaktiv telemedicin} gør det muligt for patient og læge at dele information og kommunikere i real-time. Disse informationer kan f.eks. være fra selvmonitoreringsteknologi, digitalkameraer og røntgenbilleder. Interaktiv telemedicin behøver ikke foregå i real-time, men dækker over, at lægen kommer med en respons tilbage til patienten ud fra informationer der er udvekslet. \citep{InteractiveTelemedicine}

\subsection{Telemonitorering ved hjertesvigt} 
I 2015 blev der af Center for innovativ medicinsk teknologi lavet en erfaringsopsamling i forbindelse med udbredelse af telemedicinsk hjemmemonitorering \citep{erfaringsopsamlingTelemedicin}. Her blev der foretaget et systematisk review af både dansk og engelsk litteratur, der belyser evidensen af telemedicinsk hjemmemonitorering. Dette review undersøgte litteratur om hjemmemonitorering indenfor KOL, diabetes mellitus og hjertesvigt. Konklusionen var, at evidensen var mangelfuld og der ikke tegner sig entydige tendenser, på nær ved hjertesvigt, hvor telemedicinsk hjemmemonitorering førte til reduceret dødelighed og færre indlæggelser i flere studier. Der blev desuden fundet, at telemedicinsk udstyr til hjertesvigt har en positiv effekt på livskvaliteten, mens der er modstridende resultater for de økonomiske konsekvenser. I følge erfaringsopsamlingen kan de modstridende resultater for den økonomiske konsekvens, skyldes metodiske problemer i studierne, da studierne ikke har haft dette som formål. Der er ingen studier der rapporterer om negative effekter på kliniske outcomes. \citep{erfaringsopsamlingTelemedicin}\\
%På nuværende tidspunkt er der i Region Nordjylland et projekt i samarbejde med Aalborg Universitetshospital og Aalborg Universitet, der undersøger både sundhedsøknomiske og patientnære implikationer af telemedicin til hjertesvigtpatienter, i forhold til nuværende patientforløb uden telemedicin. Projektet forløber fra 2016-2018. Det telemdicinske udstyr består af en tablet, der samler data fra eksterne måleinstrumenter og data fra spørgeskemaer. De eksterne målinger består af blodtryk, puls og vægt. Disse data sammen med data fra spørgeskemaet bliver sendt trådløst til et centralt klinisk system. Tabletten underretter desuden patienten når det er tid til at tage dagens måling. Der findes endnu ingen offentliggjorte studier omkring resultaterne. \citep{telemedicnNordjylland}\\
I et andet studie foretaget af \citet{teleprog} blev 48 hjertesvigts patienter overvåget ved telemonitorering. Overvågningen foregik ved at en sygeplejerske eller hjemmehjælper besøgte patienten 1 uge efter udskrivelse fra hospital og derefter hver måned. Der blev under disse opfølgninger foretaget et EKG, ved hjælp af en mobil EKG. Data herfra blev sendt til en database til videre analyse. Mellem hver opfølgning blev patienterne ringet op af samme sygeplejerske hver 7. eller 15. dag der kørte patienten igennem et spørgeskema for at bedømme patientens helbred. Blev alarm parametre verificeret blev en speciallæge i Kardiologi kontaktet og en behandlingsplan blev udarbejdet. \citep{teleprog} 
Forsøget viste at kun 12\% af patienterne behøvede genindlæggelse indenfor 30 dage efter udskrivelse og at den økonomiske udgift for indlæggelser gik fra 116.856 Euro, fra året før patienterne var med i telemedicin forsøget, til 40.065 Euro, året efter deltagelse i forsøget. \citep{teleprog} Da telemonitorering af hjertesvigts patienter potentielt kan gøre det muligt at foretage justeringer i den enkelte patients behandling, kan antallet af genindlæggelserne altså reduceres \citep{Inan2018}.
% kilden siger: med bærbare enheder
Fordi der findes forskellige modaliteter til at monitorere hjertefunktionen, er det derfor relevant at undersøge, hvilke af disse der vil være egnede til hjemmemonitorering.