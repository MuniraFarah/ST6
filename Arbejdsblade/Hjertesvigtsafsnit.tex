\section{Hjertesvigt}

De patofysiologiske processer i hjertesvigt er yderst komplekst. Flere forskellige hypoteser som forklarer tilstanden er fremsat. \citep{Abraham2007}
Hjertesvigt forbliver ufuldstændigt forstået af forskere og der er ikke en enkelt samlet ramme som har vist sig holdbart  \citep{Coronel2001}. 
Hjertesvigt er et syndrom som forekommer sekundært til abnormaliteter i hjertemusklens struktur, funktion eller begge. Abnormaliteteten uanset årsagen, resulterer i hjertets manglende evne til imødekomme kroppens metaboliske behov. \citep{Shah2011}\citep{Fletcher2001}\citep{Francis1998}\citep{Mudd2008} Dette syndrom er associeret med nedsat motionstolerance, dyspnø under hvile og anstrengelse og ødemer i underekstremitet. \citep{Francis1998}

Hjertesvigt er en almindelig klinisk syndrom men de patofysiologiske faktorer kan variere betydelig mellem patienterne \citep{Parmley1985}. Selvom der er mange årsager eller tilstande som kan føre til hjertesvigt, så er de mest almindelige årsager ukontrolleret hypertension, koronararteriesygdom og mitral eller aortaklap dysfunktion \citep{Fletcher2001}. Derudover ses der også sygdomme som kronisk øger belastning på hjertet, som ved tab af myocytter pga. hjerteinfarkt \citep{Shah2011} 

Et normalt fungerende hjerte er i stand til at foretage præcise justeringer i slagvolumen til at imødekomme ændringer kroppens metaboliske behov ved hvile og motion. Disse fysiologiske variationer i slagvolumen er mulige pga. af hjertets elasticitet eller kontraktilitet og er sammen med minutvolumen påvirket af fire variabler: inotropi, hjertets kontraktilitet; preload eller diastolisk fyldningsvolume; afterload eller mængden af tryk som ventriklen skal overkomme for at aortaklappen åbnes; og kronotropi eller hjerterytme. \citep{Fletcher2001}  

Der findes overordnede to typer af hjertesvigt; systolisk og diastolisk. Forskellen mellem systolisk og diastolisk dysfunktion kan bestemmes vha. ejection fraction (EF). Systolisk dysfunktion er karakteriseret med EF på mindre end 40 \%, \citep{Consensus1999}\citep{Fletcher2001} Systolisk dysfunktion er karakteriseret ved en reduceret venstre ventrikel kontraktilitet, der ofte resulterer i et dilateret hjerte, som ikke er i stand til at vedligeholde en tilstrækkelig minutvolumen. \citep{Mudd2008} \citep{Consensus1999}\citep{Fletcher2001} Hos patienter med diastolisk hjertesvigt, er volumen i den venstre ventrikel typisk normal, hvor EF er større end 50 \% \citep{Ruzumna1996},  og oftest ses hypertrofisk kardiomypati. \citep{Braunwald2013}. Hjertet er i dette tilfælde ikke i stand til at fylde sig tilstrækkeligt under diastole, hvilket fører til en nedsat slagvolumen, nedsat minutvolumen og symptomer på hjertesvigt. \citep{Ruzumna1996} \citep{Willerson1995}\citep{Fletcher2001} Generelt er det iskæmisk hjertesygdomme som er den mest almindelig årsag til systolisk dysfunktion mens venstre ventrikel hypertrofi, sekundært til hypertension eller klap abnormaliteter, udgør størstedelen ved diastolisk dysfunktion \citep{Mcalister1997}. 

Når der er en nedsat minutvolumen og sekundært nedsat gennemsnitlig arterietryk, er der en stimulering af flere neurohormonelle systemer som vedligeholder hæmodynamisk stabilitet \citep{Mccance1998}. Her kan kan blandt andet nævnes en stimulering af det sympatiske nervesysstem (SNS) til at udskille adrenalin og noradrnelina som medfører til en øget perifer vaskulær modstand, hjerterytme og kontraktilitet \citep{Fletcher2001}. Redistrubering af blod flow som resulat af SNS stimulering nedsætter renal perfusion, som medfører til en aktivering af renin-angiotensin-aldosteron-system (RAAS). Dette vil efterfølgende medføre til vasokonstriktion og retention af vand og salt. \citep{Mccance1998} Renal retention af salt og vand sammen med en øget perifer vaskulær modstand fører til en øget preload og afterload, som medvirker til pulmonal og vaskulær ødemer og skaber symptomer som er karakteristisk ved hjertesvigt. \citep{Mccance1998}

En aktiverering af det SNS og RAAS vil på længere sigt medføre remodellering af ventriklerne og yderligere myokardieskader, hvorved en ond cirkel opstår. \citep{Braunwald2013} Ventrikulær remodellering er en kompleks patofysiologisk process som manifesterer sig som ændringer i størrelse, form og funktion af hjertet. Eksempelvis vil en skade på hjertet medføre til  myocytnekrose \citep{Cohn2000}. Dette ses blandt andet hos patienter med koronararteriesygdom som har haft en eller flere myokardieinfarkt \citep{Page1971}. Som et forsøg på at vedligholde minutvolumen efter tab af kontraktil væv, vil de overlevende myocytter forlænge og hypertrofere. Som ventriklerne forstørres så vil ventiklerne væggene bliver tyndere og begynde at dilatere, hvilket resulterer i dilateret kardiomyopati. \citep{Cohn2000} Remodellering kan forekomme efter akut hjerteinfarkt eller globalt som resultat af kardiomypati. Hvis det ikke behandles, kan ændringen i venstre ventrikels geometri resultere i ændringer i muskelfunktionen og kan føre til udvikling af sekundær mitral insufficiens, som efterfølgende fører til yderligere forringelse af minutvolumen. \citep{Gheorghiade1998}

Initielt er disse kompensatoriske mekanismer gavnlige, men over tid vil disse dog stresse hjertet yderligere og forværre hjertesvigt \citep{Mccance1998} som i sidste ende øger sandsynligheden for organsvigt eller en forværring i den klinisk prognose\citep{Mudd2008}. Her kan hjerterytmen nævnes der som regel er øget ved hjertesvigt, formentlig som reflektion af en stigning af cirkulerende katekolaminer og sympatisk tone. Stigning i hjerterytmen er vigtigt til at vedligeholde minutvolumen ved lav slagvolumen. Under visse omstændigheder, kan overdrevet  hjerterytme være skadeligt. For eksempel har hjerterytme en direkte forbindelse til myokardiel ilt forbrug, så jo højere hjerterytme, jo større behov for koronar blood flow er der. Derudover med øget hjerterytme, forkortes den diastoliske tid under koronar flow. Derfor kan en stinging i hjerterytme ikke kun øge efterspørgelsen, men kan potentielt reducere forsyningen, og kan dermed have en negativ effekt for patienten. \citep{Parmley1985}
%%%%%


%Den mest almindelig årsag til højresidet hjertesvigt er venstresidet hjertesvigt. Dog er der en antal vigtige forhold som fører til pulmonal hypertension og højre sidet hjertesvigt ved fravær af en direkte involvering af venstre ventrikel, hvoraf mitral stenose er et eksempel på disse forhold. Emfysem og kronisk obstruktiv lunge sygdom, især i forbindelse med rygning, er hyppige årsager til et forhøjet pulmononat tryk og højre sidet hjertesvigt. Gentagende pulmonar emboli kan også være årsag til syndromet. \citep{Parmley1985} 


% Disse modforanstaltninger inkluderer, retention af salt og vand af nyrerne, stimulering af kroppens organer med neurohormoner og aktivering af intracellulære signaleringskaskader i hjertet og vaskulaturen, der ændrer form og funktion af i celler og organer. Disse kompensatoriske vil initielt kunne nulstille den reducerede hjertemuskles ydeevne, men bliver en medkonspirator i sygdomsprocessen, Efter hjertefunktionen nedsættes, vil kroppen kompensere ved at øge stimulering af det sympatiske nervesystem og RAAS. Disse øger hjerterytmen og intensiteten af hjertemusklens kontraktion og mængden af væske retention, som et forsøg på at vedligeholde minutvolumen [4]. fra mudd
%Der sker et antal af hormonelle ændringer ved hjertesvigt. Dette er en stigning i cirkulerende katekolaminer som skyldes en stigning i det sympatiske tone og udskillelsen af katekolaminer fra adrenal medulla. [22] Cirkulerende katekolaminer menes at vedligheolde kardiovaskulære funktioner ved en nedsat kontraktilitet. RAAS aktiveres også og medfører blandt andet vasokonstriktion, og kan medvirke til en stigning i det systemiske vaskulære modstand. Desuden faciliterer det sympatisk outflow og øger det cirkulerende katekolaminer. Derudover fører det til en yderligere retention af salt og vand. [23]

%Hyppige årsager til hjertesvigt: Overbelastning af tryk er som regel forårsaget af systemtisk hypertension eller obstruktion, såsom ved aortaklapstenose. Volumen overload kan forekomme ved aorta- eller mitralregurgitation.  Forlænget tryk eller volume overload fører til ændringer i myokardiekontraktilitet, som for det meste synes at være irreversible. [1]

%De hyppigste årsager bidrager til hjertesvigt ved tilstedeværelsen af koronararterie sygdom. Tab muskler pga. myokardieinfarkt og efterfølgende ardannelse. Reduktion i hjertets pumpeevne og ventrikulær reserve som naturligt kommer efter en reduktion af kontraktile elementer. Iskæmi er også en vigtig måde at reducere funktion. Ved anigna eller akut iskæmi, er der en abrupt reduktion, da iskæmiske muskler hurtig mister evnen til at kontraktere. [4] fra parma
%Når muskler tabes under infarkt, er der en øget loading af de resterende normale muskler. På samme måde så vil det tryk eller volume overload evt føre til en irreversibel reduktion i kontraktilitet. Et aspekt hos patienter med koronararteriesygdom er en nedsat ventrikuær compliance som forekommer i proces af ardannelse efter infakrt. Yderligere under akut anginca pectoris, ser det ud til at ventriklerne bliver stivre, formentlig pga iskæmi og relaksation abnormaliteter. [7] fra parma