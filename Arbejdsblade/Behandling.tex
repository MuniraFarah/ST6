\subsection{Behandling}
Udredningen samt behandlingen af hjertesvigt er tilrettelagt som et pakkeforløb \citep{sstpakke} for at sikre en effektiv diagnosticering, behandling og rehabilitering af patienten \citep{sstpakke}. Eftersom der er en variation i den bagvedliggende årsag til hjertesvigt, individualiseres behandlingen af patienterne \citep{TSchroeder2016}. Som et led i behandlingen tilbydes alle patienterne en farmakologisk og non-farmakologisk behandling, afhængigt af sygdommens karakter. Formålet er at eliminere eller forhindre progression af lidelsen samt øge livskvaliteten \citep{sstpakke}.\\
Den non-farmakologiske behandling består af modifikation af risikofaktorerne ved livsstilsændringer, herunder rygeophør, vægttab, motion samt diabetes og hypertension \citep{Hjerteinsufficiens}.\\
Den farmakologisk behandling kan være en kombination af forskellige præparater og består typisk af ACE-hæmmere, diuretica, beta-blokkere og aldosteron antagonister \citep{TSchroeder2016}\citep{sstpakke}. Nogle patienter henvises til en mere invasiv revaskularisering, i form af perkutane metoder (PCI) eller by-pass-operation, men dette er ikke anbefalet uden tilstedeværelsen af angina pectoris \citep{sstpakke}. Patienter i NYHA klasserne III og IV har ligeledes gavn af en elektromekanisk behandling i form af en Implantable Cardioverter Defibrillator (ICD). Herudover overvejes hjertetransplantation til patienter under 60 år, med terminal hjertesvigt \citep{edok}\citep{Hjerteinsufficiens}.

\subsection{Rehabilitering og opfølgning}
Efter behandlingen opfølges der jævnligt på patienterne ved et ambulant besøg. Her er fokus på, at retningslinjerne bliver fulgt samt en vurdering af sygdomsstadie, compliance, rehabilitering og en kontrol af risikofaktorerne blodtryk, kolesterol, og blodsukker \citep{Hjerteinsufficiens}\citep{edok}. Oftest er der ca. 3-4 kontroller om året \citep{edok}. I et studie foretaget af \citet{Michalsen1998}, blev det estimeret, at over halvdelen af indlæggelserne potentielt kunne afværges, idet hovedparten af indlæggelserne vurderedes at være compliance relaterede \citep{Michalsen1998,Hjerteinsufficiens}. Studiet konkluderede, at uddannelse, opfølgning og rehabilitering af patienten var væsentlige for at reducere antallet af indlæggelser og svære tilfælde. 

\textbf{Prognose:}\\
En variation i behandlingen resulterer i en variation af prognosen. Dog afhænger en god prognose i høj grad af en tidlig diagnose, en god compliance, rehabiliteringsforløb, den bagvedliggende årsag samt komorbiditeten, da patienter med hjertesvigt typisk præsenteres med et kompleks af årsager \cite{sstpakke}. I Danmark ligger 1 års mortaliteten på omkring 10-20\% og afhænger af blandt andet af alder, køn, den bagvedliggende sygdom samt mængden af skade på hjertet \cite{7,sundhed}. Desuden kan NYHA klasserne fungere som en yderligere prognostisk indikator \cite{Hjerteinsufficiens}, hvor prognosen for funktionsklassernes I til IV er hhv. 5-10 \%, 10-20 \%, 30 \% og 50 \% \cite{sundhed}. En stor andel af patienterne opleves endvidere at forblive i samme funktionsklasse over en lang række år, mens det menes at medicin bidrager til at patienten kan vende tilbage til lavere niveau. Dog virker komorbiditerne som katalysatorer, da der ses en dobbelt så høj mortalitet hos patienter med iskæmisk hjertesygdom, nedsat nyrefunktion og diabetes \cite{Hjerteinsufficiens}.
