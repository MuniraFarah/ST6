\section{Behandling}

Der skal tages forbehold for en række faktorer når en hjertsyg patient behandles, hvorfor behandlingen individualiseres udfra parametre som alder, funktionsniveau, anden sygdom etc.. \cite{basis}. Som et led i behandlingen tilbydes patienten både pharmakologisk og non-pharmakologisk tilgang, hvor de hhv. indebærer en modifikation af bg5e
ee2b. Disse har til formål at reducere


Risikofaktorer bør modificeres. Behandlingen
individualiseres un d e r hensyn til symptomer, alder,
funktionsniveau, an den sygdom og patientens ønsker \cite{}.

Antitrombotisk medicin reducerer morta lite ten og
forekomsten a f nye iskæmiske episoder og er basisbeh
an dling ved alle former for iskæmisk hjertesygdom

Hvis patienten ikke kan reddes, benyttes palliativ tilgang

Revaskularisering i form a f p e rk u tan k o ro n a r -in
tervention (PCI) eller “ballonbehandling” er i dag den hyppigst anvendte metode. Ved hjælp

af et ledekateter og en guidewire indføres et ballonkateter
i koronararterien. Kateteret føres hen til det snævre
sted, fyldes m ed kontrast og under flere atmosfærers tryk
dilateres arterien. Som regel anlægges efterfølgende en
stent – et lille perforeret metalrør, der holder det dila
terede k o ro n a rkar u d spænd t. Recidivhyppigheden afhænger
a f karrets størrelse, beskaffenhed, stenttype og
genetiske forhold hos patienten. De seneste år ha r man
indført medicinafgivende stents, hvor stenten er dækket
med cytostatika, der frigives lokalt og hæmme r celledelingen
af de glatte muskelceller. Det forhindrer en del af
restenoserne.
