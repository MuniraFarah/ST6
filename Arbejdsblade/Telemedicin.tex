\section{Telemonitorering}
Den demografiske udvikling i Danmark er, at den ældre befolkning bliver større, mens den arbejdsdygtige andel bliver mindre. Da ældre mennesker oftere har kroniske sygdomme, vil antallet af personer med kroniske sygdomme i Danmark dermed også stige. Dette vil gøre det nødvendigt med en bedre ressourceudnyttelse i den offentlige sektor. Det forventes at telemedicin kan bidrage hertil. \citep{erfaringsopsamlingTelemedicin}\\
Telemedicin er brugen af telekommunikation til at levere sundhedsydelser på afstand. Dette er gjort muligt, da adgangen til telekommunikationsudstyr som mobiltelefoner og computere er øget.
Telemedicinsteknologi kan overordnet deles op i tre kategorier: fjernmonitorering, store-and-forward og interaktiv telemedicin. Disse kan desuden kombineres. \citep{InteractiveTelemedicine}\\
\\
\textbf{Fjernmonitorering} er teknologier som mobiltelefoner, der kan bruges til at monitorere patienter med langvarige sygdomme, ved at overføre kliniske data fra patienten derhjemme til en læge. I forhold til rutinemæssige ambulante besøg, gør dette lægen i stand til hurtigere at ændre behandlingen, hvis patientens data giver grundlag for dette. \citep{InteractiveTelemedicine}\\
\\
\textbf{Store-and-forward} systemer gemmer data, så det kan blive analyseret på et senere tidspunkt. Dette kan f.eks. være billeder taget af en radiograf, der gemmes elektronisk, og senere kan tilgås af en læge. \citep{InteractiveTelemedicine}\\
\\
\textbf{Interaktiv telemedicin} gør det muligt for patient og læge at dele information og kommunikere i real-time. Disse informationer kan f.eks. være fra selvmonitoreringsteknologi, digitalkameraer og røntgenbilleder. Interaktiv telemedicin behøver ikke foregå i real-time, men dækker over, at lægen kommer med en respons tilbage til patienten ud fra informationer der er udvekslet. \citep{InteractiveTelemedicine}\\
\\
I 2012 blev der lavet en national handlingsplan for udbredelsen af telemedicin, og efterfølgende i 2013 udarbejdede regeringen, KL og Danske Regioner en fælles-offentlig strategi for digital velfærd, der blandt andet indeholder en række initiativer for udbredelsen af telemedicin i hele landet. \citep{erfaringsopsamlingTelemedicin}\\
\\
\subsection{Studier og forsøg med telemedicin til hjemmemonitorering af hjertesvigt} 
Til evaluering af telemedicinske løsninger, bruges ofte den europæiske model: Model for Assessment of Telemedicine (MAST). Denne bygger på Medicinsk Teknologi Vurdering (MTV), men er specielt udviklet til telemedicin. MAST indeholder følgende 7 domæner der bør evalueres \citep{erfaringsopsamlingTelemedicin}:
\begin{itemize}
    \item Domæne 1 – Helbredsproblem og teknologi: Beskrivelse af helbredsproblemet, teknologien og de tekniske karakteristika
    \item Domæne 2 – Sikkerhed: Klinisk sikkerhed for patienten og personalet og teknisk sikkerhed (driftssikkerhed)
    \item Domæne 3 – Klinisk effekt: Effekt på dødelighed, sygelighed, livskvalitet og forbrug sundhedsydelser
    \item Domæne 4 – Patientens perspektiver: Tilfredshed og accept, patient empowerment, evne til at benytte teknologien og adgang til behandling
    \item Domæne 5 – Økonomiske aspekter: Økonomiske konsekvenser og gevinster baseret på effekterne fra de øvrige domæner
    \item Domæne 6 – Organisatoriske aspekter: Ændring i ressourcer og arbejdsgange og personalets holdning
    \item Domæne 7 – Sociokulturelle, etiske og juridiske aspekter
\end{itemize}
-
Ud fra dette blev der af Center for innovativ medicinsk teknologi i 2015 lavet en erfaringsopsamling i forbindelse med udbredelse af telemedicinsk hjemmemonitorering \citep{erfaringsopsamlingTelemedicin}. Her blev der gennemgået både dansk og engelsk litteratur, der belyser evidensen af telemedicinsk hjemmemonitorering. Litteraturgennemgangen undersøgte litteratur om hjemmemonitorering indenfor KOL, diabetes mellitus og hjerteinsufficiens (hjertesvigt). Konklusionen var, at evidensen var mangelfuld og der ikke tegner sig entydige tendenser, på nær ved hjertesvigt, hvor telemedicinsk hjemmemonitorering førte til reduceret dødelighed og færre indlæggelser i flere studier. Der blev desuden fundet, at telemedicinsk udstyr til hjertesvigt har en positiv effekt på livskvaliteten, mens der er modstridende resultater for de økonomiske konsekvenser. I følge erfaringsopsamlingen kan de modstridende resultater for den økonomiske konsekvens, skyldes metodiske problemer i studierne, da studierne ikke har haft dette som formål. Der er ingen studier der rapporterer om negative effekter på kliniske outcomes. \citep{erfaringsopsamlingTelemedicin}\\
\\
På nuværende tidspunkt er der i Region Nordjylland et projekt i samarbejde med Aalborg Universitetshospital og Aalborg Universitet, der undersøger både sundhedsøknomiske og patientnære implikationer af telemedicin til hjertesvigtpatienter, i forhold til nuværende patientforløb uden telemedicin. Projektet forløber fra 2016-2018. Det telemdicinske udstyr består af en tablet, der samler data fra eksterne måleinstrumenter og data fra spørgeskemaer. De eksterne målinger består af blodtryk, puls og vægt. Disse data sammen med data fra spørgeskemaet bliver sendt trådløst til et centralt klinisk system. Tabletten underretter desuden patienten når det er tid til at tage dagens måling. Der findes endnu ingen offentliggjorte studier omkring resultaterne. \citep{telemedicnNordjylland}

Pga. forværring af deres tilstand bliver patienter med hjertesvigt bliver hyppigt indlagt på hospitalet. Det skyldes bla. dårlig terapeutisk adhærens og en utilstrækkelig opfølgning af patienten efter udskrivelse. Den første måned efter  udskrivelse har patienter den højeste risiko for genindlæggelse, hvor op til 24\% genindlægges som følge af forværringer. \cite{teleprog} 

I et forsøg foretaget af .. blev 48 hjertesvigts patienter overvåget ved telemonitorering. Overvågningen foregik ved at en sygeplejerske eller hjemmehjælper besøgte patienten 1 uge efter udskrivelse fra hospital og derefter hver måned. Der blev under disse opfølgninger foretaget et EKG, ved hjælp af en mobil EKG (HeartView P-12/8). Data herfra blev sent til en database til videre analyse. Mellem hver opfølgning blev patienterne ringet op af samme sygeplejerske hver 15. eller 7. dag der kørte patienten igennem et questionnaire for at bedømme patientens helbred. Blev alarm parameter varaficeret blev en speciallæge i Kardiologi kontaktet og en behandlingsplan blev udarbejdet. \cite{teleprog} 
Forsøget viste at kun 12\% af patienterne behøvede genindlæggelse indenfor 30 dag efter udskrivelse og at den økonomiske udgift for indlæggelser gik fra 116.856 Euro, fra året før patienterne var med i telemedicin forsøget, til 40.065 Euro, året efter det deltog i forsøget. \cite{teleprog}

%En fordel ved telemedicin, er at patienterne ikke skal bruge ressourcer på at tage på sygehuset til eventuelle kontroller. Det vil også kunne spare sundhedsvæsenet penge, da hjemmemonitorering giver patienterne mulighed for selv at lave undersøgelser, der hidtil har krævet fagpersonale på et sygehus. Der er desuden en samfundsøkonomisk gevinst, idet hjemmemonitorering giver mulighed for hurtigere intervention, og dermed et bedre behandlingsforløb.