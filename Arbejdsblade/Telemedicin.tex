Telemedicin er brugen af telekommunikation til at levere sundhedsydelser på afstand. Dette er gjort muligt, da adgangen til telekommunikationsudstyr som mobiltelefoner og computere er øget.
Telemedicinsteknologi kan overordnet deles op i tre kategorier: fjernmonitorering, store-and-forward og interaktiv telemedicin. Disse kan desuden kombineres.\\
\\
\textbf{Fjernmonitorering} er teknologier som mobiltelefoner, der kan bruges til at monitorere patienter med langvarige sygdomme, ved at overføre kliniske data fra patienten derhjemme til en læge. I forhold til rutinemæssige ambulante besøg, gør dette lægen i stand til hurtigere at ændre behandlingen, hvis patientens data giver grundlag for dette.\\
\\
\textbf{Store-and-forward} systemer gemmer data, så det kan blive analyseret på et senere tidspunkt. Dette kan f.eks. være billeder taget af en radiograf, der gemmes elektronisk, og senere kan tilgås af en læge.\\
\\
\textbf{Interaktiv telemedicin} gør det muligt for patient og læge at dele information og kommunikere i real-time. Disse informationer kan f.eks. være fra selvmonitoreringsteknologi, digitalkameraer og røntgenbilleder. Interaktiv telemedicin behøver ikke foregå i real-time, men dækker over, at lægen kommer med en respons tilbage til patienten ud fra informationer der er udvekslet.\\
\\
En fordel ved telemedicin, er at patienterne ikke skal bruge ressourcer på at tage på sygehuset til eventuelle kontroller. Det vil også kunne spare sundhedsvæsenet penge, da hjemmemonitorering giver patienterne mulighed for selv at lave undersøgelser, der hidtil har krævet fagpersonale på et sygehus. Der er desuden en samfundsøkonomisk gevinst, idet hjemmemonitorering giver mulighed for hurtigere intervention, og dermed et bedre behandlingsforløb.