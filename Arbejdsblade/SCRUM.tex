\section{SCRUM}
\textit{}

SCRUM er en organisatorisk tilgang til udførelsen af et produkt, primært softwareudvikling. Ved brug af SCRUM fremsættes en overordnet liste, en backlog, af planlagt indhold og egenskaber, disse prioriteres, og det videre arbejde planlægges i mindre segmenter.
Forud for en forklaring af de forskellige faser i SCRUM, bør de forskellige roller indeholdt i SCRUM forklares. Der findes tre primære roller tilhørende SCRUM: Ejeren, udviklerne, og SCRUM masteren.


\textbf{Ejeren} fastsætter produktets omfang, backloggen, i form af indhold og egenskaber, og ejeren står med ansvaret for dette. Disse skal også rangeres, og mens dette kan overlades til udviklerne, skal disse stadig forhøre sig hos ejeren for eventuelle ændringer til en prioriteret liste.\\
\textbf{Udviklerne} står for udarbejdelse af produktet. Disse holdes ansvarlige for at bidrage med fremskredne dele af produktet, ift. Sprint deadlines. Set ift. SCRUM, inddeles udviklerne ikke i mindre hold eller roller. Alle er lige ansvarlige for at et potentielt udgivelsesparat produkt er færdigt ved Sprint deadline.\\
\textbf{SCRUM masteren} er ansvarlig for optimal brug af SCRUM hos både ejeren og udviklerne. Denne hjælper ejeren med organisation af produktets backlog, således opgaverne med højest prioritet, hvad enten det er praktikalitet eller andet, udføres først. Yderligere hjælper SCRUM masteren udviklerne med at få en god forståelse for SCRUM, således opgaverne der udføres bliver gjort på bedste vis.\\


SCRUM i sig selv inddeles i forskellige dele og faser. Forud samt under udvikling af et produkt, findes en backlog for produktet. I backloggen findes alle dele af produktet. Backloggen fastsættes før arbejde på produktet påbegyndes, og denne ændres løbende som krav for produktet ændres eller udføres. Ejeren står for udarbejdelse og opdatering af backloggen, og dette gøres ofte i samarbejde med udviklerne. Efterfølgende for udarbejdelse af en produkt backlog holdes et Sprint planlægningsmøde. På baggrund af produkt backloggen samt planlægningen for den opkommende Sprint, udarbejdes en Sprint backlog, hvor arbejde som anses af høj prioritet for et udgivelsesparat produkt uddelegeres til udviklerne. Denne skal være detaljeret, således det er muligt på det daglige SCRUM møde at beskrive ændringer for planlagt fremskridt af Sprint. En Sprint er omtrent 2-4 uger i længde, og det arbejde som planlægges udført ændres ikke undervejs. Der arbejdes ikke på baggrund af hvad der er planlagt udført i Sprint backloggen, men udelukkende på baggrund af den afsatte mængde tid. Efterfølgende for en Sprint udarbejdes en opdatering for Sprint backloggen, og en ny Sprint påbegyndes. Undervejs i Sprint afholdes daglige møder, hvor det kommende døgn diskuteres for udvikling af produktet. Samtidig bruges det daglige møde som en synkronisering for udviklerne.


Overordnet kilde: Scrum.org → Skal have skrevet lidt flere på, som scrumguide.org og scrumalliance.org osv.
