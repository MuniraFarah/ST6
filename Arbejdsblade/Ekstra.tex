\section{Genindlæggelser}

Hjertesvigtforløbet kan, ifølge \citet{Gheorghiade2009}, inddeles i følgende fire faser: en tidlig fase, en hospitalsfase, en præ-udskrivelsesfase, og en tidlig post-udskrivelsesfase, hvoraf sidstnævnte varer de første få uger efter udskrivelsen.\\
Den tidlige fase foregår som regel på skadestuen, hvor patienten stabiliseres og årsagen til sygdommen identificeres og behandles. I hospitalsfasen er patienten indlagt og behandles og monitoreres for eventuelle myokardieskader. Ved præ-udskrivelsesfasen oplever størstedelen af patienterne at have fået en lindring i symptomerne, at den bagvedliggende årsag til anfaldet er håndteret, samt at patienten har fået en plan om efterbehandlingsforløbet med klare instruktioner herom.

Post-udskrivelsesfasen er der hvor majoriteten af patienterne, svarende til 80-85 \%, oplever en forværring af deres symptomer og renale funktioner grundet hæmodynaiske og neurale abnormaliteter, hvilket er medvirkende til den høje mortalitets- og genindlæggelsesrate. \citep{Gheorghiade2009} Disse er på henholdsvis 10 \% og 20 \% efter udskrivelsen, men stiger til henholdsvis 20 \% og 30 \% efter 3-6 måneder \citep{GFonarow2007}. Ifølge \citet{Keenan2008} genindlægges hver fjerde patient efter 30 dage og næsten 50 \% af genindlæggelserne er hjerte-relaterede \citep{Gheorghiade2009}. %Genindlæggelserne skyldes oftest væskeophobning i lungerne \citep{VLueder2012,Gheorghiade2009,Inan2018}. 

Forværringen, som opstår på trods af at patienten er under behandling, kan tilskrives en lang række faktorer. Herunder dårlig compliance, dårlig patientuddannelse, eller udløsende faktorer, såsom iskæmi, hypertension og atrieflimmer \citep{Gheorghiade2009}. Disse medfører en stigning i frekvensen af hjertesvigtsrelaterede genindlæggelser \citep{Murray2009}. Der er generelt gjort en stor indsats i at forbedre morbiditet og mortalitet i denne patientgruppe over de seneste årtier \citep{VBetihavas2013}. Dog er hjertesvigt, til trods for de tiltag der er gjort, den mest omkostningsrige sygdom i USA og Europa med en relativ dårlig prognose \citep{}. Dette skyldes blandt andet patienternes komorbiditet, alder \citep{JOyanguren2016}, men også at der ikke reageres på forværringen i tide, da advarselssignalerne og symptomerne oftest er uspecifikke eller ikke indset af patienten selv \citep{VConraads2011}, hvilket i sidste ende medfører genindlæggelserne. Forværringen er ligeledes medvirkende til at der opstår myokardieskader og en nedsat renal funktion, hvilke er bidragende faktorer til prognosen \citep{Gheorghiade2009}. Desuden er der en sammenhæng mellem antallet af episoder med forværring samt genindlæggelsen og en dårligere prognose \citep{JOyanguren2016,VConraads2011,VLueder2012}, særligt i den ældre del af befolkningen \citep{VLueder2012}. Foruden de hjerterelaterede konsekvenser for den ældre befolkning, medfører indlæggelserne også ikke-hjerterelaterede tilfælde, herunder fald, infektioner og konfusion \citep{VBetihavas2013}.%, som heller ikke er komplikationsfrie.

Det er altså estimeret, 80-85 \% af patienterne med kronisk hjertesvigt genindlægges grundet forværringer der opstår 
kort tid efter den første udskrivelse fra hospitalet og at 50 \% af disse genindlæggelser er hjertesvigtsrelaterede. Eftersom forværringen over tid, blandt andet, giver anledning til irreversible skader på myokardiet, en dårligere prognose, en højere genindlæggelsesrate og flere omkostninger. Derfor er der brug for at patienterne reagerer tidligst muligt på forværringerne, hvilket er forsøgt igennem patientuddannelse. Dog viser det sig, at genindlæggelsesraten har været relativ konstant igennem det seneste årti \citep{Inan2018}, hvorfor der er behov for en indsats så snart de første tegn viser sig. Da forværringen sker i patientens eget hjem, er en form for hjemmemonitorering en oplagt mulighed for at sætte ind så snart der varsler tegn på forværring og ihvertfald inden forværringen har forvoldt irreversibel skade \citep{}. 