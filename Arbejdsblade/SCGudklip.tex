

\begin{comment}
\gls{SCG}SCG som et bærbart målingsystem har den primære fordel at det gøres muligt at opsamle data løbende i dagligdagen, som potentielt kan hentes i ethvert miljø. Dette gør det muligt at vurdere en persons kardiovaskulære præstation under forskellige miljømæssige forhold \citep{inan2015}. \\



I et studie foretaget af \cite{Inan2018} bedømmes hjertesvigtstilstand hos ambulante og indlagte patienter med en bærbar SCG. Resultaterne af denne undersøgelse indikerer at det er muligt at vurdere hjertesvigttilstanden ved at sammenligne strukturen af et SCG-signal før og efter motion\citep{Inan2018}.


\textit{Af de inkluderede studier var der ikke . Dog er der gjort nogle tilnærmelser for at give et indtryk af hvad der på nuværende tidspunkt har en veldokumenteret effekt.}

- En lukket søgning har vist at man, ved at benytte et simpelt accelerometer, kan udtrække et SCG signal.
- Dog benyttes SCG 


Shafiq


I et studie foretaget af Shafiq et al., blev en kombination af EKG og SCG sensorer benyttet til at lave en automatiseret lokalisering af peak detektioner. Ved at skelne mellem peaksne, ville intervaller såsom PEP, LVET og QS2 kunne kvantificeres (som tilsammen kaldes systolisk time intervals (STI)). Det er generelt svært på grund af variationer imellem patienterne og støjen i SCG signalet. Dette system, som kan mointorere de hæmodynamiske parametre kontinuert, kan give en indikation af den underliggende hjertesygdom selv før symptomerne manifestere sig. SCG signalet blev opsamlet analogt og blev derefter digitaliseret med Nidaq.
Deres effektmål er, hvor godt deres automatiske algoritmer kan identificere peaksne. Hvis deres målte peak er uden for en thresh-hold for den rigtige peak, kaldes denne misclassified peak.
Desuden er AO peaken den letteste at identificere grundet dens høje magnitude i relation til de andre.


In this paper, we propose a scheme that automatically identifies the location of desired SCG peaks that are
required to calculate STI. The approach is based on obtaining a rough initial estimate of AO and AC peaks by
formulating a template from the ensemble average of few initial beats. This rough estimate is then employed
to obtain finer estimate by detecting the peaks in the sliding template. For each incoming beat segment, a new
sliding template is formulated by the ensemble averaging of the previous few beat segments. The undesired distortions
and noise effects are minimized in this process such that the peaks are easier to detect. Further, sliding
template aids in avoiding error propagation in case of erroneous peak detection.
R








----------------------

Rienzo

I et studie foretaget af Rienzo et al, blev PEP og LVET undersøgt for deres stabilitet i forskellige fysiske aktiviteter. Ved brug af en kombination af EKG og SCG (Magic-SCG) blev disse STI parametre målt som er kendte for hjertets kontraktilitet. De fandt, at denne metode/ dette udstyr til at måle disse parametre var gode og robuste overfor støj.


---------------------------

Inan

Kan bruges da:
-	Viser forskel mellem compensated og decompensated på baggrund af SCG og databehandling

We measure the seismocardiogram signal using
a custom wireless patch26 mounted at the midsternum that
includes an ultralow noise triaxial accelerometer to capture
the 3 components of the signal: dorso-ventral, head-to-foot,
and lateral accelerations, as shown in Figure 1.



------------------------

Etamadi og inan

Denne gruppe har undersøgt BCG til at måle vibrationer fra kroppen, hvorfra hjerte-mekaniske parametre kan udledes (CO, BP, contractility). Samtidig er BCG et billigt alternativ i hele befolkningen.
Gruppen har sammenlignet to typer af systemer;
-	Elektroder på sternum: Måler STI (PEP, ændringer i CO)
-	Ur: Måler pulse transmit time (PTT)  altså den tid det tager fra AO til det bliver opfanget af uret. Herfra beregnes BP
Desuden kommer de ind på alogritmer til at minimere bevægelsesartifaktor osv.
BCG blev målt sammen med EKG
SCG blev målt og bruger også PPG (prototype)


Jeg tænker slet!



\textbf{Etemadi og Inan 2018}\\
\textbf{Inan et al 2018}\\
\textbf{Rienzo et al 2013}\\
\textbf{Sahoo et al 2017}\\
\textbf{Shafiq et al 2016}\\

Konkluder at det er på forskningsbasis
Fælles for de inkluderede studier var, at ... Målingerne blev foretaget i et kontrolleret miljø, hvortil fremtidige studier bør evaluer e denne teknologi i patientens hjem med henblik på hjememmonitorering. Desuden kan fremgangsmåden i fremtiden testes til at følge patientens tilstand og deres respons til farmakologiske interventioner. \citep{Inan2018} 
----------------------

\section{Problemformulering}
- Opsummering af hele problemanalysen + probleafgrænsningen. Men ret på tallene først

\begin{center}
\textit{Problemformuleringen...}
\end{center}
\end{comment}