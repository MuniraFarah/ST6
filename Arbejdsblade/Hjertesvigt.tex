\section{Hjertesvigts noter}

Tackling heart failure in the twenty-first century

Hæmodynaisk og neurohormon

Individer med hjertesvigt udvikler som regel symptomer gradvist og de bliver mindre aktive og oplever flere episoder af akut hjertesvigt også kaldt for dekompensation.


Efter hjertefunktionen nedsættes, vil kroppen kompensere ved at øge stimulering af det sympatiske nervesystem og RAAS. Disse øger hjerterytmen og intensiteten af hjertemusklens kontraktion og mængden af væske retention, som et forsøg på at vedligeholde minutvolumen [4]. 

Patalogiske stressorer såsom pressure overload stimulerer processer som kan føre til hypertrofisk kardiomypati.

Regulering af hjertemusklens energi forsyning og metabolisme er tæt reguleret. Denne regulering bliver kompromitteret ved hjertesvigt, som kan føre til en tilstand med ineffektivitet og en manglende energiforsyning.

Heart failure by braunwald 
Der forekommer flere patogeniske mekanismer som opererer ved hjertesvigt. Disse inkluderer øget hæmodynamisk overload, iskæmisk relateret dysfunktion, ventrikulær remodellering, overdreven neurohormonel (neurohumoral) stimulering, abnormal myocyt calcium cyklus, overdreven eller utilstrækkelig proliferation (vævsvækst) af det extracellulære matric, accelereret apoptose (celledød) og genetisk mutationer. 

Hæmodynamisk model: En abnormalitet i myokardie funktionen som medvirker til at hjertemusklens pumpeevne er nedsat og ikke er tilstrækkelig til at imødekomme de metaboliserende væv under normal aktivitet [25]. Ved hjertesvigt ses der ved en stigning af hæmodynamisk load, at der sker en reduktion af kontraktilitetn af hjertemuskulaturen [26]. Ved ventrikulær remodellering ses der hæmodynamiske ændringer, som er meget normalt hos patienter med kronisk dysfunktion af den ventrikulære pumpe. Hos patienter med diastolisk hjertesvigt, er volumen i den venstre ventrikel typisk normal, men væggen er fortykket. Hos patienter systolisk hjertesvigt er venstre ventrikel typisk dilateret.

Myokardie nekrose øger udskillelsen af væksthormoner i bindevæv, som resulterer i formation af nye fibroblaster. Når denne proces er utilstrækkelig, som efter en infarkt, sker der en fortynding af den ventrikulære væg. Den øgede syntese af ekstracellulære matriks øge myokardie stivhed ved pressure overload hypertrofi og reducerer rytme ved ventrikulær afslapning såvel som ved kontraktion [33]. Fibrose kan blive stimuleret ved at long-term aktivering af renin-angiotensin-aldosterone systemet [34]. 

Ved akut hjertesvigt, aktiveres det sympatiske system som medfører til en øget kontraktilitet. Der forekommer vasokonstriktion som øger blodtrykket og hjælper med perfusion af de vitale organer. En aktiverern af det SNS og RAAS vil dog på længere sigt medfører remodellering af ventriklerne og yderligere myokardie skader, hvorved en ond cirkel opstår. 


In search of new therapeutic targets and strategies forheart failure: recent advances in basic science

Kronisk hjertesvigt er et kompleks klinisk syndrom som forekommer sekundært abnormaliteter i hjertemusklens struktur, funktion eller begge som hæmmer hjertemusklens evne til at fylde eller udmpumpe blod. Hyppige årsager inkluderer sygdomme som kronisk øger hjertemusklens arbejdsbelastning, som ved tab af myocytter pga. myokardie infarkt eller tryk overload pga. hypertension. Hjertemusklens respons til sådanne påvirkninger medfører kompleks remodellering, som initielt kan være adaptivt men eventuelt kan udvikle sig til kontraktil dysfunktion, ventrikulær dilation og arytmier. 

Heart failure by krum

Hjertesvigt er som regel associeret med en struktruel abnormalitet af hjertet. Den oprindelige skade kan være pludselige og indlysende (f.eks. myokardieinfarkt) eller uklar (f.eks. langvarig hypertension). Når skaden er sket, starter en proces som først er kompenserende men efterfølgende maladaptive mekanismer. 

Kompensatoriske mekansimer som aktiveres ved hjertesvigt inkluderer: øget ventrikulær preload, eller Frank-Starling mekanisme, med ventrikulær dilation og volumen øgelse [31], perifer vasokonstriktion, som initielt vedligholder perfusion til de vitale organer; myokardiel hypertrofi til at bevare hjertemusklens spænding mens hjertet dilaterer; renal og vand retention til at øge ventrikulær preload; og en start af det adrnerge nervesystem som øger hjerterytmen og kontraktile funktioner [32]. Disse processer er hovvedsagligt styret af forskellige neurohormonelle vasokonstriktor systemer, inkulderende RAAS, det adrenerge system, og ikke-osmostiske udskillelse af arginine-vasopressin [33].

En stigning i wall stress sammen med aktivering af neurohormoner fremmer patologisk ventrikulær remodelering; denne process er tæt knyttet til progression af hjertesvigt.

A general theory of acute and chronic heart failure

De patofysiologiske processer i hjertesvigt er yderst komplekst. Flere forskellige hypoteser som forklarer tilstanden er fremsat. Disse inkluderer hæmodynamik, neurohormonel, muskel og diastolisk hjerte svigts teori [1]. Hjertesvigt forbliver ufuldstændigt forstået af forskere og der er ikke en enkelt samlet ramme der har slået tidstesten (stood the test of time) [2]. 

Forskellige hypoteser er ikkke nedskrevet men står i artiklen. 

I akut og destabliseret kronisk hjertesvigt er der som regel en reduceret venstre ventrikel slag volumen. Tidligere mekanismer er normalt fremkaldt/invoked, såsom takykardi, inotropisk stimulation og Frank-Starling mekanimse, for forbedre minutvolumen [31,32]. 

Hjertets principielle funktion er at sørge for en tilpas minutvolumen til at forsyne væv til deres metabolisme både ved hvile og under aktivitet. Minutvolumen er bestemt af både hjerterytmen og slagvolumen [40].

Rehospitalization for heart failure by gheorgiade

Hjertesvigt er ikke en sygdom, men en manifestation af forskellige kardiel og ikke kardiel abnormaliteter [3=gheo2009]. 

Heart failure by jama
Hjertesvigt udvikles når hjertets pumpevne er defekt. Systolisk hjertesvigt er hjertemusklens manglende evne til at pumpe nok blod fra ventriklerne til at forsyne kroppens behov. Diastolisk hjertesvigt er et resultat af hjertesmuklens manglende enve til at slappe af mellem hjerteslagene, som medfører til backup af blod i ventriklerne og i blodårene. Både systolisk og diastolisk hjertesvigt kan medføre ødemer i lunger og resten af kroppen. Hjertemusklen forsøger at kompenserer for defekten ved at litagere eller blive hypertrofisk. 

Heart failure with preserved ejection fraction:
pathophysiology, diagnosis, and treatment by borlaug. 

Diastolisk hjertesvigt hvor tilstedeværelsen af normal systolisk venstre ventrikel ydeevne med diastolisk venstre ventrikel dysfunktion, som består af en forlænget isovolumetrisk venstre ventrikel afslapning, langsom fyldning af venstre ventrikel og øget diastrolisk venstre ventrikel stivhed [1-4]. 

Ved fravær af endokardiel eller perikardiel sygdom, resulterer diastolisk venstre ventrikel dysfunktion fra øget myokardiel stivhed. Det som regulerer den diastoliske stivhed er den ekstracellulær matric og cardiomyocytter. 

Pathophysiology of Congestive Heart Failure by Parmley

Hjertesvigt er en almindelig klinisk syndrom men de patofysiologiske faktorer kan varierer betydlig mellem patienterne. 

Hyppige årsager til hjertesvigt:
Overbelastning af tryk er som regel forårsaget af systemtisk hypertension eller obstruktion, såsom ved aortaklapstenose. Volumen overload kan forekomme ved aorta- eller mitralregurgitation.  Forlænget tryk eller volume overload fører til ændringer i myokardie kontraktilitet, som for det meste synes at være irreversible. [1]

Et eksempel på tab af muskler som en årsag til hjertesvigt er hos patienter med koronararteriesygdom som har haft en eller flere myokardiinfarkt. Dette tab af muskler nedsætter hjertemusklens pumpeevne og kan føre til de samme irreversible ændringer i de resterende normale myokardie. Ved akut myokardie infarkt, vil 40 \% eller mere af myokardiet resulterer i kardiogent shock. [2]

En nedsat kontraktilitet af hjertemusklen viser sig fx ved tilstand såsom volum og tryk overload og kardiomypati.

Begrænsing fyldning af hjertet forekommer ved konstriktiv perikarditis, hjertetamponade eller tilstande hvor stiv ventrikulær kamre begrænser fyldningen. Eksempeler på sidsnævnte indkluder endomykardiel fibrose eller en hypertroferet ventrikel. 

Koronararterie sygdom er den mest almindelig årsag til hjetesvigt og udgør 2/3 af alle patienter med severe class IV hjertesvigt. De hyppigste årsager bidrager til hjertesvigt ved tilstedeværelsen af koronararterie sygdom. Tab muskler pga. myokardie infarkt og efterfølgende ardannelse. Reduktion i hjertets pumpeevne og ventrikulær reserve som naturligt kommer efter en reduktion af kontraktile elementer. Iskæmi er også en vigtig måde at reducere funktion. Ved anigna eller akut iskæmi, er der en abrupt reduktion, da iskmpiske muskler hurtig mister evnen til at kontraktere. [4] Når muskler tabes under infarkt, er der en øget loading af de resterende normale muskler. På samme måde så vil det tryk eller volume overload evt føre til en irreversibel reduktion i kontraktilitet. Et aspekt hos patienter med koronararteriesygdom er en nedsat ventrikuær compliance som forekommer i proces af ardannelse efter infakrt. Yderligere under akut anginca pectoris, ser det ud til at ventriklerne bliver stivre, formentlig pga iskæmi og relaksation abnormaliteter. [7]

Den mest almindelig årsag til højresidet hjertesvigt er venstresidet hjertesvigt. Dog er der en antal vigtige forhold som fører til pulmonar hypertension og højre sidet hjertesvigt ved fravær af en direkte involvering af venstre ventrikel, hvoraf mitral stenose er et eksempel på disse forhold. Emfysem og kronisk obstruktiv lunge sygdom, især i forbindelse med rygning, er hyppige årsager til en forhøjet pulmononar tryk og højre sidet hjertesvigt. Gentagende pulmonar embolig kan også være årsag til syndromet. 

Reduktion af minutvolumen er en af de principielle manifestationer ved hjertesvigt. 

Der sker et antal af hormonelle ændringer ved hjertesvigt. Dette er en stigning i cirkulerende katekolaminer som skyldes en stigning i det sympatiske tone og udskillelsen af katekolaminer fra adrenal medulla. [22] Cirkulerende katekolaminer menes at vedligheolde kardiovaskulære funktioner ved en nedsat kontraktilitet. RAAS aktiveres også og medfører blandt andet vasokonstriktion, og kan medvirke til en stigning i det systemiske vaskulære modstand. Desuden faciliterer det sympatisk outflow og øger det cirkulerende katekolaminer. Derudover fører det til en yderligere retention af salt og vand. [23]

Selvom kompensatoriske mekanismer kan hjælpe med at vedligeholder hjertefunktion, kan det potentielt overskyde og have en skadelig effekt. Retention af salt og vand  som forekommer ved nedsat minutvolumen og renal perfusion kan føre til en overdrevet stigning i preload. 

Det systemtiske vaskulære restians er ofte for høj ved hjertesvigt. Som minutvolumen nedsattes , øges den vaskulære modstand gennem neurohormonelle refleks mekanismer som beskrevet foroven. Denne øget modstand kan yderligere øge modstand ved udpumpning af blod, reducere minutvolumen og derfor føre til en tilstand hvor minutvolumen er lavere og det systemiske vaskulær modstand er højere en optimalt. [25]

Hjerterytmen er som regel øget ved hjertesvigt, formentlig som reflektion af en stigning af cirkulerende katekolaminer og sympatisk tone. Stigning i hjerterytme er bestem vigtigt til at vedligeholde minutvolumen ved lav slagvolumen. Under visse omstændigheder, kan overdret  hjerterytme være skadeligt. For eksempel har hjerterytme en direkte forbindelse til myokardiel ilt forbrug, så jo højere hjerterytmet, jo større et behov for koronar blood flow er der. Derudover med øget hjerterytme, forkortes den diastoliske tid under koronar flow. Defor en stinging i hjerterytme kan ikke kun øge efterspørgelsen, men kan potentielt reducere forsyningen, og kan dermed have en negative effekt for patient med koronararteriesygdommen.

The management of congestive heart failure by mcalister

Patofysiologisk er hjertesvigt karakteriseret ved hjertets manglende evne til at pumpe blod med en hastighed svarende til de metaboliserende væs behov og/eller kun at gøre det ved forhøjet påfyldningstryk [1]. Dette resulterer i et komplekst klinisk syndrom med multipel organs involvering, der manifesterer sig i en lang række symptomer og fysiske tegn.

Systolisk vs diastolisk:
Kompensatorisk aktivering af neuroendokriner ser ud til at forekomme stort set hos patienter med systolisk dysfunktion. Systolisk dysfunktion er karakteriseret ved en reduceret venstre ventrikel kontraktilitet og manifisterer i en reduceret left ventricular ejection fraction (LVEF). Ved diastolisk dysfunktion er der en svækket ventrikulær afslapning og øget compliance i diastolse. Som resultat har disse patienter som regel en normal LVEF men en øget venstre atriel tryk og kamretryk øges markant som respons til en stigning i venstre ventrikel volumen. 

Selvom de fleste patienter besidder egenskaber fra både systolisk og diastolisk dysfunktion, så er systolisk svækkelse den mest dominerende defekt hos størstedelen. Den rapporeret prævalens af dominerende diastolisk dysfunktion vairer fra 10-30 i litteraturen. [10] 

Generelt er det iskæmisk hjertesygdomme som er den mest alindelig årsag til systolisk dysfunktion men venstre ventrikel hypertrfo, sekundært til hypertension eller klap abnormaliteter, udgør størstedelen ved diastolisk dysfunktion.  

Systolisk dysfunktion er som regel associeret med myokardie infarkt og dilation af venstre ventrikel. 

Congestive Heart Failure: Understanding the Pathophysiology and Management by Fletcher


Hjertesvigt er defineret som en tilstand der er resultat af abnormaliteter i myokardie funktionen. Abnormaliteteten uanset årsagen, resulterer hjertetes manglende evne til at levere nok iltet blod til at imødekomme kroppens metaboliske behov. Når højre og venstre ventrikel ikke fungerer som pumper, opstår der pulmonar og systemisk venøs hypertension, som resulterer i syndrom hjertesvigt. Denne syndrom er associeret med nedssat motionstolerance, dyspnø under hvile og anstrengelse og ødemer i under ekstremer. [Francis1998]

Et normalt fungerende hjerte er i stand til at lave præcise justeringer i slagvolumen til at imødekomme ændringer kroppens metaboliske behov ved hvile og motion. Disse fysiologiske variationer i slagvolumen er mulige fordi at hjertets elasticitet som resulterer i en optimal ventrikel tømning (minutvolumen) uden at øge hjertes ilt behov eller variation en arteriel tryk. Elasticiteten eller kontraktilitetn af hjertet og minutvolumen er påvirket af fire variabler: inotropi, hjertets kontraktilitet; preload eller diastolisk fyldningsvolume; afterload eller mængden af tryk som ventriklerne har behov for at aortaklappen åbnes (systemic vascular resistance);og kronotropi eller hjerterytme.  

Selvom der er mange årsager eller tilstand som kan føre til hjertesvigt, så er den mest almindelig årsag ukontrolleret hypertension, koronararteriesygdom og mitral eller aortaklap dysfunktion.  Hjertesvigt repræsenterer en tilstand hvor minutvolumen ikke kan imødekomme kroppens metaboliske behov. 

Forskellen mellem systolisk og diastolisk dysfunktion kan bestemmes vha. ejection fraction (EF). (Her andens Doppler ekkokardiografi). Systolisk dysfunktion er karakteriseret med EF på indre end 40 \% og er defineret som en nedsat kontraktil kraft af myokardium og resulterer ofte i en tynd og dilateret hjertemuskel som ikk er i stand til at vedliholde en tilstrækelig minutvolumen. [Consensus recommendations for heart failure 1999] Systolisk dysfunktion er den mest almindelige årsag til hjertesvigt og er oftest relateret til skade forårsaget af myokardieinfarkt [abraham 2000]. 

Diastolisk dysfunktion er en svækket evne til diastolisk fyldning af venstre ventrikel og er sekudært til tab af muskel fiber elasticitet. Det er oftest associeret med hypertension over længere tid. Op til 40 \% af patienter med symptomatisk hjertesvigt har diastolisk hjertesvigt. [willerson 1995]. De mekanismer der er ansvarlig for diastolisk dysfunktion, uanset normal systolisk funktion, er nedsat ventrikulær compliance eller nedsat stivhed og svækket ventrikulær afslapning. Svækkelse i diastolisk fyldning af venstre ventrikel kan resulterer fra iskæmi, hypertrofi, reduktion af beta-adrenerge tone eller øget myokardie bindevæv. [Ruzumna, ghorghiade 1996]. Patienter med preserved systolic funktion (EF> 50\%) som har symptomer på hjertesvigt er klassificeret ved at have diastolisk dysfunktion. [ruzumna et al]. Over tid, vil det øget arbejde af venstre ventrikel medføre hypertorfi af msuklen, som yderligere nedsætter kontraktilitetn af muskel fibrene og resulterer i en svækket relaksation. Det resulterende non-compliant hypertroferet venstre ventrikel er ikke i stand til at fylde sig tilstrækkelig og føre til en nedsat slagvolumen, nedsat minutvolumen og symptomer på hjertesvigt. [ruumna et al] Efterfølgende vil den lave minutvolumen stimulere kompensatoriske neurohormonel system som øger det cirkulerende blod volumen og fyldningstryk og resulteer i yderligere pulmonar congestion.

Kompensatoriske mekanismer: Når der er en nedsat minutvolumen og sekundært nedsat gennemsnitlig arterietryk, er der en stimulering af flere neurohormonelle systemer som vedligeholder hæmodynamisk stabilitet. Initielt er disse kompensatoriske mekanismer gavnlige; dog over tid vil de stresse hjertet yderligere og forværre hjertesvigt. [mccance 1998] 

Som respons til nedsat minutvolumen, baroreceptorer inden i aortabuen og karotid stiumlere det sympatiske nervesysstem til at udskille adrenaling og noradrnelina som medfører til en øget perifer vaskulær modstand, hjerterytme og kontraktilitet. 

Til gengæld, redistrubering af blod flow som resulat af SNS stimulering nedsætter renal perfusion, som medfører til en aktiveres af RAAS. Efterfølgende medfører til vasokonstriktion og retention af vand og salt. [mccance 1998] Renal retention af salt og vand sammen med en øget perifer vaskulær modstand fører til en øget preload og afterload, som yderligere medvirker til pulmunar og vaskulær congestion og skabe symptomer som er karakteristisk ved hjertesvigt. [mccance 1998]

Kronisk sympatisk stimulering, som forekommer ved hypertension, han have negative effekt på hjertet som ventikuær remodellering [connolly]. 
Ventrikulær remodellering er en kompleks patofysiologisk process som manifesterer sig som ændringer i størrelse, form og funktion af hjertet. Som resultat af skade på hjertee, dør myocytter som ikke kan regenerere. Som et forsøg til at vedligholde minutvolumen efter tab af kontraktil væv, vil de overlevende myoceytter forlænge og hypertrofere. Som ventriklerne forstørres så vil ventiklerne væggene bliver tyndere og begynde at dilaterer, hvilket reulster i dilateret kardiomyopati. [cohn ferrar 200] Remodellering kan forekomme efter akut hjerteinfarkt eller globalt som resulatt af kardiomypati. Hvis det ikke behandles, kan ændringen i vesntre ventrikels geometri resulterer i ændringer i muskel funktion og føre til udvikling af sekundær mitral insufficiens, som efterfølgende føre til yderligere forringelse af minutvolumen. [gehorgiahde 1998]


