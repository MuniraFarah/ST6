\subsection{Klassificering af hjertesvigt og udredning}
Da sen diagnose af hjertesvigt giver en øget mortalitet, er det vigtigt at diagnosticere patienter tidligt og få dem i tidlig behandling. I løbet af behandlingsforløbet skal patienter til rutinemæssig kontrol hos egen læge for at sikre at sygdommen ikke forværres og at patienten dermed får den rette behandling. \citep{heartfailure}\\
Der bruges flere systemer til at klassificere hjertesvigt, der er baseret på enten symptomer eller progression \citep{heartfailure}. Formålet er at finde ud af hvilken behandling og prognose patienten har. I Danmark inddeles hjertesvigts patienter primært i NYHA (New York Heart Association) funktionsklasser \citep{DCS}:

\begin{itemize} 
\item NYHA I: Der opleves ingen symptomer eller begrænsninger under almindelig fysisk aktivitet.
\item NYHA II: Der opleves ingen symptomer under hvile eller under lettere fysisk aktivitet, men lav grad af åndenød, træthed, hjertebanken ved moderat fysisk aktivitet som trappegang til 2. sal, havearbejde, støvsugning, bære tunge ting.
\item NYHA III:  Ingen symptomer i hvile, men blot ved let fysisk aktivitet som påklædning eller gang i fladt terræn giver udmattelse, åndenød, evt. hjertebanken eller brystsmerter.
\item NYHA IV: Der opleves symptomer i hvile, som forværres ved fysisk aktivitet
\end{itemize}

Patienter med hjertesvigt får oftest stillet diagnosen gennem kontakt med læge i forbindelse med at de oplever de førnævnte symptomer. Patienten bliver her henvist til en række undersøgelser med henblik på at undersøge hjertet. \citep{heartfailure}\\
Der bliver altid taget et elektrokardiogram, EKG, der benyttes til at sænke eller styrke mistanken om hjertesvigt. Resultatet kan desuden udelukke akutte tilstande som akut myokardie infarkt. Derudover er ekkokardiografi essentielt i diagnosen af hjertesvigt, da undersøgelsen giver en visualisering af hjertet, hvor tegn på blodprop og hjerteklapfejl kan ses og pumpefunktion kan vurderes. \citep{heartfailure}\citep{DCS}