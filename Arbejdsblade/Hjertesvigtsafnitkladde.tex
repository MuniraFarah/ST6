\section{Hjertesvigtsafsnits Kladde}

\section{Hjertesvigt}

%Hjertesvigt forbliver ufuldstændigt forstået af forskere og der er ikke en enkelt samlet ramme der har slået tidstesten (stood the test of time) \citep{Coronel2001}. 
Hjertesvigt er defineret som en tilstand der er resultat af abnormaliteter i hjertets struktur, funktion eller begge. Abnormaliteteten uanset årsagen, resulterer i hjertets manglende evne til imødekomme kroppens metaboliske behov. \citep{Shah2011} \citep{Fletcher2001} \citep{Francis1998} \citep{Mudd2008} Når højre og venstre ventrikels pumpefunktion ikke fungerer som pumper, opstår der pulmonal og systemisk venøs hypertension, som resulterer i syndrom hjertesvigt. Denne syndrom er associeret med nedsat motionstolerance, dyspnø under hvile og anstrengelse og ødemer i underekstremitet. \citep{Francis1998}

Et normalt fungerende hjerte er i stand til at foreage præcise justeringer i slagvolumen til at imødekomme ændringer kroppens metaboliske behov ved hvile og motion. Disse fysiologiske variationer i slagvolumen er mulige fordi at hjertets elasticitet som resulterer i en optimal ventrikel tømning uden at øge hjertes ilt behov eller variation i arteriel tryk. Elasticiteten eller kontraktilitetn af hjertet og minutvolumen er påvirket af fire variabler: inotropi, hjertets kontraktilitet; preload eller diastolisk fyldningsvolume; afterload eller mængden af tryk som ventriklerne har behov for at aortaklappen åbnes; og kronotropi eller hjerterytme. \citep{Fletcher2001}  

%
Hyppige årsager inkluderer sygdomme som kronisk øger hjertemusklens arbejdsbelastning, som ved tab af myocytter pga. myokardie infarkt eller tryk overload pga. hypertension. Hjertemusklens respons til sådanne påvirkninger medfører kompleks remodellering, som initielt kan være adaptivt men eventuelt kan udvikle sig til kontraktil dysfunktion, ventrikulær dilation og arytmier. \citep{Shah2011}
Hjertesvigt er som regel associeret med en struktruel abnormalitet af hjertet. Den oprindelige skade kan være pludselige og indlysende (f.eks. myokardieinfarkt) eller uklar (f.eks. langvarig hypertension). Når skaden er sket, starter en proces som først er kompenserende men efterfølgende maladaptive mekanismer. krum
De patofysiologiske processer i hjertesvigt er yderst komplekst. Flere forskellige hypoteser som forklarer tilstanden er fremsat. Disse inkluderer hæmodynamik, neurohormonel, muskel og diastolisk hjerte svigts teori [1]. 
Hjertesvigt er en almindelig klinisk syndrom men de patofysiologiske faktorer kan variere betydelig mellem patienterne \citep{Parmley1985}. 

Selvom der er mange årsager eller tilstand som kan føre til hjertesvigt, så er den mest almindelig årsag ukontrolleret hypertension, koronararteriesygdom og mitral eller aortaklap dysfunktion \citep{Fletcher2001}.

Individer med hjertesvigt har flere symptomer tilfælles; fatigue, åndenød og væskeretention. 

Der findes overordnede to typer af hjertesvigt; systolisk og diastolisk. Systolisk er defineret som en tilstand med nedsat kontraktilitet, der ofte resulterer i et dilateret hjerte  som ikke er i stand til at vedligeholde en tilstrækkelig minutvolumen. 

Systolisk hjertesvigt er en tilstand med dilateret hjerte og kontraktil svigt mens dette ikke er tilfældet ved diastolisk hjertesvigt, hvor der oftest ses hypertrofisk kardiomypati. \citep{Mudd2008}
Hos patienter med diastolisk hjertesvigt, er volumen i den venstre ventrikel typisk normal, men væggen er fortykket. Hos patienter systolisk hjertesvigt er venstre ventrikel typisk dilateret. \citep{Braunwald2013}

Forskellen mellem systolisk og diastolisk dysfunktion kan bestemmes vha. ejection fraction (EF). Systolisk dysfunktion er karakteriseret med EF på mindre end 40 \% og er defineret som en tilstand med en nedsat kontraktilite, der ofte resulterer i en tynd og dilateret hjertemuskel som ikke er i stand til at vedligeholde en tilstrækkelig minutvolumen. \citep{Consensus1999} \citep{Fletcher2001} Systolisk dysfunktion er den mest almindelige årsag til hjertesvigt og er oftest relateret til skade forårsaget af myokardieinfarkt \citep{Abraham1998} \citep{Fletcher2001}.

Diastolisk dysfunktion er en svækket evne til diastolisk fyldning af venstre ventrikel og er sekundært til tab af muskel fiber elasticitet. Det er oftest associeret med hypertension over længere tid. Op til 40 \% af patienter med symptomatisk hjertesvigt har diastolisk hjertesvigt. \citep{Willerson1995} \citep{Fletcher2001} 

De mekanismer der er ansvarlig for diastolisk dysfunktion, uanset normal systolisk funktion, er nedsat ventrikulær compliance eller nedsat stivhed og svækket ventrikulær afslapning. Svækkelse i diastolisk fyldning af venstre ventrikel kan resultere fra iskæmi, hypertrofi, reduktion af beta-adrenerge tone eller øget myokardie bindevæv. Patienter med preserved systolic funktion (EF> 50\%) som har symptomer på hjertesvigt er klassificeret ved at have diastolisk dysfunktion. \citep{Ruzumna1996} Over tid, vil det øget arbejde af venstre ventrikel medføre hypertorfi af musklen, som yderligere nedsætter kontraktilitetn af muskel fibrene og resulterer i en svækket relaksation. Det resulterende non-compliant hypertroferet venstre ventrikel er ikke i stand til at fylde sig tilstrækkelig og føre til en nedsat slagvolumen, nedsat minutvolumen og symptomer på hjertesvigt.  \citep{Ruzumna1996} Efterfølgende vil den lave minutvolumen stimulere kompensatoriske neurohormonel system som øger det cirkulerende blod volumen og fyldningstryk og resultater i yderligere pulmonar congestion \citep{Fletcher2001}.


Kompensatoriske mekanismer: Når der er en nedsat minutvolumen og sekundært nedsat gennemsnitlig arterietryk, er der en stimulering af flere neurohormonelle systemer som vedligeholder hæmodynamisk stabilitet. Initielt er disse kompensatoriske mekanismer gavnlige; dog over tid vil de stresse hjertet yderligere og forværre hjertesvigt. \citep{Mccance1998}

%
Hjertesvigts tilstanden udløser nogle modforanstaltninger; retention af salt og vand af nyrerne, stimulering af kroppens organer med neurohormoner og aktivering af intracellulære signaleringskaskader i hjertet og vaskulaturen, der ændrer form og funktion af i celler og organer. Disse kompensatoriske vil initielt kunne nulstille den reducerede hjertemuskles ydeevne, men bliver en medkonspirator i sygdomsprocessen, som i sidste ende øger sandsynligheden for organsvigt eller en forværrende klinisk prognose\citep{Mudd2008}. 
Efter hjertefunktionen nedsættes, vil kroppen kompensere ved at øge stimulering af det sympatiske nervesystem og RAAS. Disse øger hjerterytmen og intensiteten af hjertemusklens kontraktion og mængden af væske retention, som et forsøg på at vedligeholde minutvolumen [4]. fra mudd
Der sker et antal af hormonelle ændringer ved hjertesvigt. Dette er en stigning i cirkulerende katekolaminer som skyldes en stigning i det sympatiske tone og udskillelsen af katekolaminer fra adrenal medulla. [22] Cirkulerende katekolaminer menes at vedligheolde kardiovaskulære funktioner ved en nedsat kontraktilitet. RAAS aktiveres også og medfører blandt andet vasokonstriktion, og kan medvirke til en stigning i det systemiske vaskulære modstand. Desuden faciliterer det sympatisk outflow og øger det cirkulerende katekolaminer. Derudover fører det til en yderligere retention af salt og vand. [23]


Som respons til nedsat minutvolumen, baroreceptorer inden i aortabuen og karotid stiumlere det sympatiske nervesysstem til at udskille adrenaling og noradrnelina som medfører til en øget perifer vaskulær modstand, hjerterytme og kontraktilitet \citep{Fletcher2001}. 

Til gengæld, redistrubering af blod flow som resulat af SNS stimulering nedsætter renal perfusion, som medfører til en aktiveres af RAAS. Efterfølgende medfører til vasokonstriktion og retention af vand og salt. \citep{Mccance1998} Renal retention af salt og vand sammen med en øget perifer vaskulær modstand fører til en øget preload og afterload, som yderligere medvirker til pulmunar og vaskulær congestion og skabe symptomer som er karakteristisk ved hjertesvigt. \citep{Mccance1998}

Kronisk sympatisk stimulering, som forekommer ved hypertension, han have negative effekt på hjertet som ventikuær remodellering \citep{Connolly2000}. 
Ventrikulær remodellering er en kompleks patofysiologisk process som manifesterer sig som ændringer i størrelse, form og funktion af hjertet. Som resultat af skade på hjertee, dør myocytter som ikke kan regenerere. Som et forsøg til at vedligholde minutvolumen efter tab af kontraktil væv, vil de overlevende myoceytter forlænge og hypertrofere. Som ventriklerne forstørres så vil ventiklerne væggene bliver tyndere og begynde at dilaterer, hvilket reulster i dilateret kardiomyopati. \citep{Cohn2000} Remodellering kan forekomme efter akut hjerteinfarkt eller globalt som resulatt af kardiomypati. Hvis det ikke behandles, kan ændringen i vesntre ventrikels geometri resulterer i ændringer i muskel funktion og føre til udvikling af sekundær mitral insufficiens, som efterfølgende føre til yderligere forringelse af minutvolumen. \citep{Gheorghiade1998}

%fortynding af den ventrikulære væg
Ved akut hjertesvigt, aktiveres det sympatiske system som medfører til en øget kontraktilitet. Der forekommer vasokonstriktion som øger blodtrykket og hjælper med perfusion af de vitale organer. En aktiverern af det SNS og RAAS vil dog på længere sigt medfører remodellering af ventriklerne og yderligere myokardie skader, hvorved en ond cirkel opstår. \citep{Braunwald2013}
Perifer vasokonstriktion, som initielt vedligholder perfusion til de vitale organer; myokardiel hypertrofi til at bevare hjertemusklens spænding mens hjertet dilaterer; renal og vand retention til at øge ventrikulær preload; og en start af det adrnerge nervesystem som øger hjerterytmen og kontraktile funktioner \citep{Rundqvist1997}. 

%%%%%
Hyppige årsager til hjertesvigt:
Overbelastning af tryk er som regel forårsaget af systemtisk hypertension eller obstruktion, såsom ved aortaklapstenose. Volumen overload kan forekomme ved aorta- eller mitralregurgitation.  Forlænget tryk eller volume overload fører til ændringer i myokardie kontraktilitet, som for det meste synes at være irreversible. [1]

Et eksempel på tab af muskler som en årsag til hjertesvigt er hos patienter med koronararteriesygdom som har haft en eller flere myokardiinfarkt. Dette tab af muskler nedsætter hjertemusklens pumpeevne og kan føre til de samme irreversible ændringer i de resterende normale myokardie. Ved akut myokardie infarkt, vil 40 \% eller mere af myokardiet resulterer i kardiogent shock. \citep{Page1971}

En nedsat kontraktilitet af hjertemusklen viser sig fx ved tilstand såsom volum og tryk overload og kardiomypati \citep{Parmley1985}.

Begrænsing fyldning af hjertet forekommer ved konstriktiv perikarditis, hjertetamponade eller tilstande hvor stiv ventrikulær kamre begrænser fyldningen. Eksempeler på sidsnævnte indkluder endomykardiel fibrose eller en hypertroferet ventrikel \citep{Parmley1985}. 

De hyppigste årsager bidrager til hjertesvigt ved tilstedeværelsen af koronararterie sygdom. Tab muskler pga. myokardie infarkt og efterfølgende ardannelse. Reduktion i hjertets pumpeevne og ventrikulær reserve som naturligt kommer efter en reduktion af kontraktile elementer. Iskæmi er også en vigtig måde at reducere funktion. Ved anigna eller akut iskæmi, er der en abrupt reduktion, da iskmpiske muskler hurtig mister evnen til at kontraktere. [4] fra parma
Når muskler tabes under infarkt, er der en øget loading af de resterende normale muskler. På samme måde så vil det tryk eller volume overload evt føre til en irreversibel reduktion i kontraktilitet. Et aspekt hos patienter med koronararteriesygdom er en nedsat ventrikuær compliance som forekommer i proces af ardannelse efter infakrt. Yderligere under akut anginca pectoris, ser det ud til at ventriklerne bliver stivre, formentlig pga iskæmi og relaksation abnormaliteter. [7] fra parma

Den mest almindelig årsag til højresidet hjertesvigt er venstresidet hjertesvigt. Dog er der en antal vigtige forhold som fører til pulmonar hypertension og højre sidet hjertesvigt ved fravær af en direkte involvering af venstre ventrikel, hvoraf mitral stenose er et eksempel på disse forhold. Emfysem og kronisk obstruktiv lunge sygdom, især i forbindelse med rygning, er hyppige årsager til en forhøjet pulmononar tryk og højre sidet hjertesvigt. Gentagende pulmonar emboli kan også være årsag til syndromet. \citep{Parmley1985} 

Selvom kompensatoriske mekanismer kan hjælpe med at vedligeholder hjertefunktion, kan det potentielt overskyde og have en skadelig effekt. Retention af salt og vand  som forekommer ved nedsat minutvolumen og renal perfusion kan føre til en overdrevet stigning i preload. \citep{Parmley1985} 

Hjerterytmen er som regel øget ved hjertesvigt, formentlig som reflektion af en stigning af cirkulerende katekolaminer og sympatisk tone. Stigning i hjerterytme er bestem vigtigt til at vedligeholde minutvolumen ved lav slagvolumen. Under visse omstændigheder, kan overdrevet  hjerterytme være skadeligt. For eksempel har hjerterytme en direkte forbindelse til myokardiel ilt forbrug, så jo højere hjerterytmet, jo større et behov for koronar blood flow er der. Derudover med øget hjerterytme, forkortes den diastoliske tid under koronar flow. Defor en stinging i hjerterytme kan ikke kun øge efterspørgelsen, men kan potentielt reducere forsyningen, og kan dermed have en negative effekt for patient med koronararteriesygdommen. \citep{Parmley1985}

Hjertesvigt resulterer i et komplekst klinisk syndromf med multipel organs involvering, der manifesterer sig i en lang række symptomer og fysiske tegn \citep{Mcalister1997}.

Kompensatorisk aktivering af neuroendokriner ser ud til at forekomme stort set hos patienter med systolisk dysfunktion. Systolisk dysfunktion er karakteriseret ved en reduceret venstre ventrikel kontraktilitet og manifisterer i en reduceret left ventricular ejection fraction (LVEF). Ved diastolisk dysfunktion er der en svækket ventrikulær afslapning og øget compliance i diastolse. Som resultat har disse patienter som regel en normal LVEF men en øget venstre atriel tryk og kamretryk øges markant som respons til en stigning i venstre ventrikel volumen. \citep{Mcalister1997}

Selvom de fleste patienter besidder egenskaber fra både systolisk og diastolisk dysfunktion, så er systolisk svækkelse den mest dominerende defekt hos størstedelen. Den rapporeret prævalens af dominerende diastolisk dysfunktion vairer fra 10-30 i litteraturen. \citep{Goldsmith1993} 

Generelt er det iskæmisk hjertesygdomme som er den mest alindelig årsag til systolisk dysfunktion men venstre ventrikel hypertrfo, sekundært til hypertension eller klap abnormaliteter, udgør størstedelen ved diastolisk dysfunktion \citep{Mcalister1997}.  
Systolisk dysfunktion er som regel associeret med myokardie infarkt og dilation af venstre ventrikel \citep{Mcalister1997}. 

