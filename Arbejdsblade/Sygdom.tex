\chapter{Hjertesvigt - sygdom}
Hjertesvigt beskriver en tilstand hvor hjertetmusklen er svækket  og hjertets pumpeevne ikke er tilstrækkelig til at imødekomme kroppens behov. Der er overordnet to forskellige typer af hjertesvigt; systolisk og diastolisk hjertesvigt. Systolisk hjertesvigt er en tilstand med dilateret ventrikel som medfører en nedsat pumpekraft og dermed nedsat evne til at opretholde minutvolumen, blodtryk og tilstrækkelig perifer cirkulation. Diastolisk hjertesvigt definerer en tilstand med nedsat fyldning af ventriklerne og medfører forhøjet fyldnignstryk/preload \cite{Basis}. 

Hjertesvigt forekommer som resultat af en myokardiskade eller længevarende belastning af myorkadiet, hvilket medfører ændringer i hjertets struktur og geometri. Dette kan eksempelvis skyldes iskæmi eller ardannelse efter et myokardiinfarkt, ventrikelhypertrofi pga. hypertension, hjertekapsygdomme eller toksisk myokardipåvirkning fx ved længevarende alkoholindtag  eller kemoterapi.  

For at kompensere for den nedsat pumpefunktion igangssættes nogle mekanismer for at opretholde mængden af blod som cirkulerer rundt i kroppen. Her kan blandt andet nævnes at hjertes slagvolumen øges, strukturen af ventriklen ændres i form af af dilation og hypertrofi, der sker en aktivering af det sympatiske nervesystem og renin-angiotensin-aldosteron-systemet. De kompensatoriske mekanismer vil på kort sigt øge cardiac output men på længere sigt vil det belaste hjertet yderligere og forværre tilstanden. 

Den øgede belastning på hjertet forekommer som resultat af en øget pre- og afterload. Aktivering af det sympatiske system øger hjertefrekvensen og  kontraktiliteten, hvilket fører til en øget afterload. Aktivering af reninssystemet medfører væske- og natriumretention, hvilket bidrager til ventrikeldilation medfører yderligere hjertebelastning ved at hæve hjertets preload. Disse processer medfører til et øget iltforbrug i myokardiet som ikke kan imødekommes og vil på længere sigt føre til myocytnekrose. \cite{Basis}. 

Hjertemusklen påføres mere arbejde som på længere sigt vil medføre til myocytnekrose. 
%preload=hvor meget væggene strækkes når ventriklen fyldes
%afterload= det tryk der skal til for at bevæge blodet over i aorta

Det forhøjede fyldningstryk udløser sekretion af natriuretiske peptider,  der forduen at fremme udskillelsen af natrium virker kardilaterende. Koncentraionen i blodet afspejler tilstandens sværhedsgrad.  Det forhøjede fyldningstryk medfører lungestase og senere venøs stase i systemkredsløbet. 

http://vejledninger.dsam.dk/media/files/8/hjerteinsufficiens.pdf

Hjertesvigt kan være venstresidigt, højresidigt eller en kombination af begge og giver forskellige symptomer alt afhængig af hvilket område af hjertet der er påvirket. Ved venstresidigt svigt vil blodet opbhobes i lungerne 


og ved højresidigt svigt ses blodophobning i kroppens vener. Når hjertet ikke pumper ordentligt, registrers det bla i nyrerne. Disse reagerer ved at holde på vand og salt i kroppen. Det øger blodmængden og dermed blodtrykket. Dette kan for en tid øge blodtilførslen til kroppens organer, men det belastede hjerte bliver endnu mere belastet af det øgede blodvolumen, og det kan udvikle sig til en ond cirkel. https://www.sundhedsguiden.dk/da/temaer/alle-temaer/hjerte-kar-sygdomme/almindelige-hjerte-kar-sygdomme/hjertesvigt-hjerteinsufficiens/ 




Sekundær sygdom som skyldes en underliggende sygdom
-	Nekrose af myocytter gør hjertet svagere, således at slagvolumen nedsættes. 
For at kompensere for den nedsat mængde blod, øges slagvolumen og hjertefrekvens. I de tidligere stadier fungerer det fint, men på længere sigt vil det belaste hjertet.  Hjertets iltbehov øges yderligere som ikke kan kompenseres for. På længere sigt vil myocytter dø pga iltmangel. Slagvolumen nedsættes yderligere og den onde cirkel gentager sig igen. 
Kompensation foregår ved at aktivere det sympatiske nervesystem. Hvis den aktiveres for meget så hjælper den mindre og mindre. 
Preload: tryk i ventriklerne efter den fyldt og og før den kontraheres  distole. Få mere blod i hjertet og musklen udvider mere og der er større tryk, så der kommer med blod ud. Øget tryk/preload giver øget slagvolumen  frank star..

kompensation: sympaticus, øge preload, mere muskelmasse. 
Hvis man kompenserer for meget så vil der forekomme dekompenssation




Hvis hjertet i perioder ikke kan yde det, der kræves, kan det føre til væskeophobning i lungerne og medføre åndenød samt vægtøgning på grund af væskeophobning i resten af kroppen, typisk i benene. 
Hvis venstre hjertehalvdel svigter, kan hjertet have svært ved at pumpe blodet videre fra lungekredsløbet og blodet vil ophobes og føre til væske i kunderne, hvilket kaldees lungestase.
Hvis højre hjertehalvdel svigter, har hjertet problemer med at pumpe det af-iltede blod tilbage til lungerne og blodet kan ophobe sig i kroppen, så man får øget væske i vævene, for eksempel hævede ben eller væske i indre organer som tarme og leveren. 


Kongestiv hjerteinsufficiens: hvis hjerteinsufficiens er ledsagt af stasefænomener i lungekredsløb eller af perifere ødemer. ’




\cite{Martini2015}