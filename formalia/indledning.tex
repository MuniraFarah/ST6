\chapter{Indledning}\label{indledningen}
Hjertesvigt beskriver en tilstand hvor hjertemusklen er svækket og hjertets pumpeevne ikke er tilstrækkelig til at imødekomme kroppens behov \citep{TSchroeder2016}. Dette giver symptomer som nedsat motionstolerence, dyspnø under hvile og anstrengelse og ødemer i underekstremiteterne \citep{heartfailure}.\\
Prævalensen er stigende, da befolkningen i den vestlige verden bliver ældre, og behandlingen af akutte hjerte-kar-sygdomme er blevet bedre, hvilket betyder at flere mennesker lever med følger som eksempelvis hjertesvigt \citep{heartfailure}\citep{Gheorghiade2009}.\\
\\
Globalt påvirker hjertesvigt omkring 2\% af den voksne befolkning. Prævalensen afhænger i høj grad af alderen, og er mindre end 2\% for personer under 60 år, og mere end 10\% for personer over 75 år. Mortaliteten er 6-7\% per år. \citep{heartfailure}\\
I Danmark var der i 2017 3812 indrapporterede patientforløb til Dansk Hjertesvigtsdatabase. 315 af disse blev genindlagt efter fire uger, svarende til en genindlæggelsesrate på 8,3\%. Dette svarer til at mere end hver 12. genindlægges efter 4 uger efter første kontakt med sygehuset. I USA genindlægges omkring hver fjerde inden 30 dage \citep{keenan2008}. \citep{RKKP2017}\\
\\
I Danmark inddeles hjertesvigtpatienter i NYHA-klasser, der er fire klasser af hjertesvigt inddelt efter symptomer \citep{DCS}. Prognose og behandling afhænger af NYHA-klasse \citep{DCS}. Da både symptomer og årsager til hjertesvigt ofte er meget individuelle, individualiseres behandlingen desuden til hver patient \citep{sstpakke}. Behandlingen består af både farmakologisk og non-farmakologisk behandling \citep{sstpakke}. Efter udskrivelsen vil der være en opfølgende konsultation senest  to uger efter udskrivelsen \citep{Sundhedsstyrelsen2018}.\\
\\
Ligesom i resten af den vestlige verden, er den demografiske udvikling i Danmark, at den ældre befolkning bliver større, mens den arbejdsdygtige andel bliver mindre. Da prævalensen af hjertesvigt stiger med alderen, kan det antages at der i fremtiden vil være flere personer i Danmark med hjertesvigt. Dette vil gøre det nødvendigt med en bedre ressourceudnyttelse i den offentlige sektor. \citep{erfaringsopsamlingTelemedicin}\\
Dette fører til det initierende problem:\\
\\
\textit{Hjertesvigtpatienter har en høj rate af genindlæggelser på sygehuset. Hvordan kan antallet af genindlæggelser minimeres?}



%Patienter med hjertesvigt har en dårlig prognose med mange hospitalsindlæggelser og høj mortalitet på 6-7\% per år. \citep{heartfailure}\\

%- Hjertesvigt er en sygdom der rammer mange og udgør en stor sundhed- og økonomisk byrde på verdensplan.\\
% generelle tal
%- I 2017 var der 3812 tilfælde af hjertesvigt i Danmark. Efter 4 uger blev 315 af disse genindlagt, svarende til en genindlæggelsesrate på 8,3 \%. Det betyder, at mere end hver 12. blev genindlagt efter deres første tilfælde, hvilket giver en indikation af at patienten forværres efter udskrivelsen.\\

% der forventes flere og flere tilfælde i takt med at man bliver bedre til at redde folk.... Hvorfor der vil ske en stigende tendens..

%- Den demografiske udvikling i Danmark er, at den ældre befolkning bliver større, mens den arbejdsdygtige andel bliver mindre. Da ældre mennesker oftere har kroniske sygdomme, vil antallet af personer med kroniske sygdomme i Danmark dermed også stige. Dette vil gøre det nødvendigt med en bedre ressourceudnyttelse i den offentlige sektor. Det forventes at telemedicin kan bidrage hertil. \citep{erfaringsopsamlingTelemedicin}\\
