\chapter{Indledning}\label{indledningen}

- Hjertesvigt er en sygdom der rammer mange og udgør en stor sundhed- og økonomisk byrde på verdensplan.\\
% generelle tal
- I 2017 var der 3812 tilfælde af hjertesvigt i Danmark. Efter 4 uger blev 315 af disse genindlagt, svarende til en genindlæggelsesrate på 8,3 \%. Det betyder, at mere end hver 12. blev genindlagt efter deres første tilfælde, hvilket giver en indikation af at patienten forværres efter udskrivelsen.\\

% der forventes flere og flere tilfælde i takt med at man bliver bedre til at redde folk.... Hvorfor der vil ske en stigende tendens..
- Hjertesvigt beskriver en tilstand hvor hjertemusklen er svækket og hjertets pumpeevne ikke er tilstrækkelig til at imødekomme kroppens behov \citep{TSchroeder2016}. Prævalensen er stigende, da befolkningen bliver ældre, og behandlingen af akutte hjerte-kar-sygdomme er blevet bedre \citep{heartfailure,Gheorghiade2009}.\\

- Kronisk hjertesvigt påvirker omkring 2\% af den voksne befolkning på verdensplan. Prævalensen afhænger i høj grad af alderen, og er mindre end 2\% for personer under 60 år, og mere end 10\% for personer over 75 år. Patienter med hjertesvigt har en dårlig prognose med mange hospitalsindlæggelser og høj mortalitet på 6-7\% per år. \citep{heartfailure}\\

- Den demografiske udvikling i Danmark er, at den ældre befolkning bliver større, mens den arbejdsdygtige andel bliver mindre. Da ældre mennesker oftere har kroniske sygdomme, vil antallet af personer med kroniske sygdomme i Danmark dermed også stige. Dette vil gøre det nødvendigt med en bedre ressourceudnyttelse i den offentlige sektor. Det forventes at telemedicin kan bidrage hertil. \citep{erfaringsopsamlingTelemedicin}\\